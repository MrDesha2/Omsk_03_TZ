\section{Описание функций подсистемы <<Управление продажами>>}



% Документ  <<Предварительная калькуляция стоимости заказа>>} 
% \input{10_Modules/Doc_sales.tex}


% Документ  <<Заявка>>} 
% \input{10_Modules/Doc_Request.tex}

% Документ  <<Заказ>>} 
\subsection{Документ <<Заказ>>}
\label{doc:Order}
\renewcommand{\curobject}{<<Заказ>>}

\subsubsection{Описание предметной области}

Документ \curobject предназначен для регистрации производственного заказа на изготовление готовой продукции.

Документ существует в СИСТЕМЕ, необходимо добавить новую функциональность.


\subsubsection{Функциональные требования}


\point{Отображение информации о проблемах по заказу}

В форму заказа добавить новую вкладку <<Проблемы по заказу>>, в которой выводить записи для текущего заказа (если есть) из регистра <<Проблемы по заказам>>.

В таблице формы <<Проблемы по заказу>> выводить флаги из регистра сведений <<Проблемы по заказу>>:
\begin{itemize}
    \item В работе (да/нет); 
    \item Решено (да/нет)
\end{itemize}








%\subsection{Обработка <<Непрерывный план>>}
%
%\subsubsection{Функциональные требования}
%
%\point {Три разных заказа в одном раскрое}
%
%Предусмотреть, что бывает потребность выпустить в одном раскрое два заказа на одном столе и один на втором. При этом заказы на одном столе могут быть как одинаковые по длине, так и
%отличаться на заданную дельту. 
%
%
%\point {Печатная форма <<Раскрои (для выделенных)>>}
%
%При печати отчета заголовки таблицы выводить перед каждой группой раскроев.
%В заголовке группы раскроев оставить вывод информации по слоям только с требуемым количеством сырья.
%
%\begin{figure*}[!htb]
%\centering
%  \includegraphics[width=180mm, height=220mm, keepaspectratio]{10_Modules/Pics/picReportTaskGa.jpg}
%\caption{Печатная форма задания на гофроагрегат}
%\label{pic:picReportTaskGa}
%\end{figure*} 
%\FloatBarrier
%

%Регистр <<Проблемы по заказам>>
\subsection{Регистр <<Проблемы по заказам>>}
\label{reg:OrderError}

\subsubsection{Описание предметной области}

Регистр существует в системе, необходимо добавить новую функциональность.


\point{Атрибуты}


Добавить новые атрибуты.
\pc
\begin{longtable}{|p{4cm}|p{4cm}|p{8cm}|}
\hline
{\bf Наименование} & {\bf Тип данных} &  {\bf Комментарий} \endhead
    \hline
    В работе & Флаг  & Признак необходимости работы по проблеме.\\
    \hline
    Решено & Флаг  & Признак выполнения работы по проблеме.\\
    \hline
  \caption{Новые поля \curobject}
 \label{tab:Reg_OrderError}
\end{longtable}




% %Отчет <<Портфель заказов>>
% \input{10_Modules/Rep_OrderBag.tex} 