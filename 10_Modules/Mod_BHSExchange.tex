\newpage

% Обмен полностью взят из Воронежа новый ГА



% \section{Требования по организации обмена данными с системой управления гофроагрегата BHS}
% \label{sec:BHS}
\section{Требования по организации обмена данными с системой управления гофроагрегатом SYNCRO}
\label{sec:fosber}

СИСТЕМА должна обмениваться данными с системой управления гофроагрегатом SYNCRO компании Fosber.

Версия протокола обмена ''Syncro STANDARD COMMUNICATION WITH OFFICE SYSTEM ver 0.5''.

Версия сиcтемы Syncro 3.7.xx.

Обмен данными должен производиться из формы редактирования документа ''ОтчетПроизводстваЛГК''.

Сетевое подключение к SYNCRO использует сконфигурированный UDP/IP-стандарт протокол. Параметры по умолчанию для доступа стандартной связи с SYNCRO представлены в таблице \ref{tab:fosber_mainparameters}.

\scriptsize
\begin{longtable}{|p{53mm}|p{20mm}|}
\hline
{\it {\bf \parbox[c][5mm]{53mm}{\raggedright Описание}}} & {\it {\bf \parbox[c]{20mm}{\raggedrightЗначение}}} \\
\hline
\parbox[c][5mm]{53mm}{Сетевой протокол} & \parbox{20mm}{UDP/IP} \\
\hline
\parbox[c][5mm]{53mm}{Удаленный IP адрес} & \parbox{20mm}{10.151.151.?} \\
\hline
\parbox[c][5mm]{53mm}{Локальный IP адрес} & \parbox{20mm}{10.151.151.1} \\
\hline
\parbox[c][5mm]{53mm}{Маска подсети} & \parbox{20mm}{255.255.255.0} \\
\hline
\parbox[c][5mm]{53mm}{Удаленный порт} & \parbox{20mm}{60001} \\
\hline
\parbox[c][5mm]{53mm}{Локальный порт} & \parbox{20mm}{60000} \\
\hline
\caption{Параметры по умолчанию для доступа стандартной связи с SYNCRO}\label{tab:fosber_mainparameters}\end{longtable}  
\normalsize

Структура сообщения представлена в таблице  \ref{tab:fosber_message}.

\scriptsize
\begin{longtable}{|p{40mm}|p{20mm}|p{70mm}|}
\hline
{\bf \parbox[c][5mm]{40mm}{\raggedright Позиция в структуре}} & {\bf \parbox[c]{20mm}{\raggedright № байта}} & {\bf \parbox[c]{40mm}{\raggedrightКомментарий}} \\
\hline
\parbox[c][5mm]{40mm}{STX} & \parbox{20mm}{1 byte} & \parbox{70mm}{02 HEX, символ начала сообщения} \\
\hline
\parbox[c][5mm]{40mm}{COMMAND} & \parbox{20mm}{2 byte} & \parbox{70mm}{Два символа, соответствующие команде} \\
\hline
\parbox[c][10mm]{40mm}{ID (Related to Run ID 1 field)} & \parbox{20mm}{4 byte} & \parbox{70mm}{Четыре символа, представляющие идентификатор раскроя} \\
\hline
\parbox[c][5mm]{40mm}{DATA BLOCK} & \parbox{20mm}{Variable} & \parbox{70mm}{Блок данных, есть команды без блока данных} \\
\hline
\parbox[c][10mm]{40mm}{MESSAGE N} & \parbox{20mm}{2 byte} & \parbox{70mm}{Два символа, используемые для связи запроса с ответом системы SYNCRO} \\
\hline
\parbox[c][5mm]{40mm}{CHECKSUM} & \parbox{20mm}{2 byte} & \parbox{70mm}{Контрольная сумма} \\
\hline
\parbox[c][5mm]{40mm}{ETX} & \parbox{20mm}{1 byte} & \parbox{70mm}{03 HEX, символ окончания сообщения} \\
\hline
\caption{Структура сообщения}\label{tab:fosber_message}
\end{longtable}  
\normalsize

\label{secLfosber_feature}
Особенности, которые необходимо учесть:
\begin{itemize}
	\item Длина сообщения зависит от длины блока данных.
	\item Пустые поля и незаполненная часть поля должны заполняться пробелами.
	\item Каждое поле должно быть выравнено по левому краю.
	\item Текстовые поля с кодировкой на кириллице должны быть перекодированы в транслитерацию.
	\item Не допускается использование специальных символов (в т.ч. ''*'', ''№'' и пр.).
	\item Для интеграции необходимо вручную привести в соответствие справочник ''Композиции'' с таблицей  ''Board grade'' гофроагрегата ***. 
	\item Запрещается редактировать поле ''Комментарий'' в задании гофроагрегата, т.к. в нём содержится идентификатор задания СИСТЕМЫ.
\end{itemize}


*** Board grade --- строка, указывающая тип бумаги, используемой для конкретной композиции сырья. SYNCRO вводит эти данные в базу данных, которая содержит другие, персонализированные оператором, полезные для оптимизации работы продольно-резательной машины Fosber. 
Эти данные также можно визуализировать на страницах SYNCRO и распечатать по запросу на этикетке комплекта.
Кодификация композиции сырья ДОЛЖНА БЫТЬ УНИКАЛЬНОЙ и совпадать в СИСТЕМЕ и SYNCRO. Это означает, что код должен быть разным для разных профилей гофрополотна и плотности бумаги/картона.


Структура команд для передачи данных из СИСТЕМЫ в SYNCRO представлена в таблице \ref{tab:fosber_comand_send}.

\scriptsize
\begin{longtable}{|p{15mm}|p{35mm}|p{40mm}|p{70mm}|}
\hline
{\bf {\parbox[c][5mm]{15mm}{\raggedright Команда}}} & {\it {\bf \parbox[c]{35mm}{\raggedright ID}}} & {{\bf \parbox[c]{40mm}{\raggedright Блок данных}}} & {\bf \parbox[c]{70mm}{\raggedright Поведение}} \\
\hline
\parbox[c][5mm]{15mm}{PE} & \parbox{35mm}{''0000'' (ASCII 30Hex)} & \parbox{40mm}{Блок ответа.} & \parbox{70mm}{Добавление раскроя в конец списка заданий.} \\
\hline
\caption{Структура команд для передачи данных Synchro}\label{tab:fosber_comand_send}
\end{longtable}  
\normalsize

Структура ответа от SYNCRO представлена в таблице \ref{tab:fosber_answer_send}

\scriptsize
\begin{longtable}{|p{15mm}|p{35mm}|p{40mm}|p{70mm}|}
\hline
{\bf {\bf \parbox[c][5mm]{15mm}{\raggedright Команда}}} & {\it {\bf \parbox[c]{35mm}{\raggedright ID}}} & {\it {\bf \parbox[c]{40mm}{\raggedright Блок данных}}} & {\it {\bf \parbox[c]{70mm}{\raggedright Поведение}}} \\
\hline
\parbox[c][5mm]{15mm}{OK} & \parbox{35mm}{4 пробела (ASCII 20Hex)} & \parbox{40mm}{34 пробела (ASCII 20Hex)} & \parbox{70mm}{Выгрузка успешна.} \\
\hline
\parbox[c][5mm]{15mm}{CK} & \parbox{35mm}{4 пробела (ASCII 20Hex)} & \parbox{40mm}{34 пробела (ASCII 20Hex)} & \parbox{70mm}{Ошибка контрольной суммы.} \\
\hline
\parbox[c][5mm]{15mm}{DF} & \parbox{35mm}{4 пробела (ASCII 20Hex)} & \parbox{40mm}{Код ошибки (4 знака) и 30 пробелов (ASCII 20Hex)} & \parbox{70mm}{Ошибка в блоке данных.} \\
\hline
\parbox[c][5mm]{15mm}{DU} & \parbox{35mm}{4 пробела (ASCII 20Hex)} & \parbox{40mm}{34 пробела (ASCII 20Hex)} & \parbox{70mm}{Дублирование идентификаторов раскроев.} \\
\hline
\parbox[c][5mm]{15mm}{RF} & \parbox{35mm}{4 пробела (ASCII 20Hex)} & \parbox{40mm}{Код ошибки (4 знака) и 30 пробелов (ASCII 20Hex)} & \parbox{70mm}{Система не может выполнить команду.} \\
\hline
\caption{Структура команд ответа от SYNCRO}\label{tab:fosber_answer_send}
\end{longtable}  
\normalsize

Структура команд для запроса данных из СИСТЕМЫ в SYNCRO представлена в таблице \ref{tab:fosber_comand_return}.

\scriptsize
\begin{longtable}{|p{15mm}|p{35mm}|p{40mm}|p{70mm}|}
\hline
{\bf {\bf \parbox[c][5mm]{15mm}{\raggedright Команда}}} & {\it {\bf \parbox[c]{35mm}{\raggedright ID}}} & {\bf {\bf \parbox[c]{40mm}{\raggedright Блок данных}}} & {\bf \parbox[c]{70mm}{\raggedright Поведение}} \\
\hline
\parbox[c][5mm]{15mm}{SE} & \parbox{35mm}{''0000'' (ASCII 30Hex)} & \parbox{40mm}{Команда без блока.} & \parbox{70mm}{Информация о текущем состоянии.} \\
\hline
\parbox[c][5mm]{15mm}{LI} & \parbox{35mm}{Номер связки} & \parbox{40mm}{Команда без блока.} & \parbox{70mm}{Информация о раскрое (по идентификатору).} \\
\hline
\parbox[c][5mm]{15mm}{LO} & \parbox{35mm}{Индекс} & \parbox{40mm}{Команда без блока.} & \parbox{70mm}{Информация о раскрое (по индексу).} \\
\hline
\parbox[c][5mm]{15mm}{LE} & \parbox{35mm}{''0000'' (ASCII 30Hex)} & \parbox{40mm}{Команда без блока.} & \parbox{70mm}{Список идентификаторов заданий в работе.} \\
\hline
\parbox[c][5mm]{15mm}{LD} & \parbox{35mm}{''0000'' (ASCII 30Hex)} & \parbox{40mm}{Команда без блока.} & \parbox{70mm}{Список идентификаторов выполненных заданий.} \\
\hline
\parbox[c][5mm]{15mm}{VE} & \parbox{35mm}{''0000'' (ASCII 30Hex)} & \parbox{40mm}{Команда без блока.} & \parbox{70mm}{Данные буфера по выработке.} \\
\hline
\parbox[c][5mm]{15mm}{DW} & \parbox{35mm}{''0000'' (ASCII 30Hex)} & \parbox{40mm}{Команда без блока.} & \parbox{70mm}{Данные буфера по остановам.} \\
\hline
\caption{Структура команд для запроса данных из СИСТЕМЫ в SYNCRO}\label{tab:fosber_comand_return}
\end{longtable}  
\normalsize

Структура ответа от SYNCRO представлена в таблице \ref{tab:fosber_answer_return}

\scriptsize
\begin{longtable}{|p{15mm}|p{35mm}|p{40mm}|p{70mm}|}
\hline
{\bf {\bf \parbox[c][5mm]{15mm}{\raggedright Команда}}} & {\it {\bf \parbox[c]{40mm}{\raggedright ID}}} & {\it {\bf \parbox[c]{40mm}{\raggedright Блок данных}}} & {\it {\bf \parbox[c]{70mm}{\raggedright Поведение}}} \\
\hline
\parbox[c][5mm]{15mm}{OK} & \parbox{35mm}{4 пробела (ASCII 20Hex)} & \parbox{40mm}{Блок ответа.} & \parbox{70mm}{Ответ сформирован.} \\
\hline
\parbox[c][5mm]{15mm}{CK} & \parbox{35mm}{4 пробела (ASCII 20Hex)} & \parbox{40mm}{Блок ответа.} & \parbox{70mm}{Ошибка контрольной суммы.} \\
\hline
\parbox[c][5mm]{15mm}{RF} & \parbox{35mm}{4 пробела (ASCII 20Hex)} & \parbox{40mm}{Блок ответа.} & \parbox{70mm}{Система не может выполнить команду.} \\
\hline
\caption{Структура ответа от SYNCRO}\label{tab:fosber_answer_return}
\end{longtable}  
\normalsize

% ПАРАМЕТРЫ РАСКРОЕВ

\scriptsize
\begin{landscape}
Структура блока параметров раскроя для передачи из СИСТЕМЫ в SYNCRO, а также для получения информации обратно, представлена в таблице \ref{tab:fosber_block}.

\begin{longtable}{|p{25mm}|p{6mm}|p{6mm}|p{8mm}|p{6mm}|p{60mm}|p{12mm}|p{100mm}|}
\hline

\parbox[c][10mm]{20mm}{\centering Поле} & \parbox{5mm}{\centering С} & \parbox{5mm}{\centering По} & \parbox{5mm}{\centering Длина} & \parbox{5mm}{\centering Тип} & \parbox{49mm}{\centering Opti-Corrugated} & \parbox{5mm}{\centering Транслит} & \parbox{80mm}{\centering Комментарий} \\
\hline
\parbox[c][20mm]{25mm}{Run ID 1} & \parbox{10mm}{1} & \parbox{10mm}{4} & \parbox{10mm}{4} & \parbox{10mm}{N} & \parbox{40mm}{Документ ''ОтчетПроизводстваЛГК''. Таблица ''Раскрои''. Номер связки} & \parbox{11mm}{} & \parbox{89mm}{Внутренний идентификатор задания на гофроагрегате. С каждым новым заданием увеличивается на единицу. Должен быть между 1000 и 8999. Введённые вручную оператором раскрои будут с номерами между 9000 и 9999.} \\
\hline
\parbox[c][20mm]{25mm}{Run ID 2} & \parbox{10mm}{5} & \parbox{10mm}{15} & \parbox{10mm}{11} & \parbox{10mm}{A} & \parbox{49mm}{Документ ''ОтчетПроизводстваЛГК''. Таблица ''Раскрои''. ID Задания} & \parbox{11mm}{} & \parbox{89mm}{Идентификатор задания. Создаётся при планировании раскроев в СИСТЕМЕ. Символьный идентификатор раскроя, который отображается на экране системы SYNCRO.} \\
\hline
\parbox[c][22mm]{25mm}{Flute} & \parbox{10mm}{16} & \parbox{10mm}{18} & \parbox{10mm}{3} & \parbox{10mm}{A} & \parbox{49mm}{Документ ''ОтчетПроизводстваЛГК''. Таблица ''Раскрои''. План. ГруппаРаскроев. Профиль. Наименование} & \parbox{11mm}{} & \parbox{89mm}{Наименование профиля.} \\
\hline
\parbox[c][22mm]{25mm}{Board grade} & \parbox{10mm}{19} & \parbox{10mm}{54} & \parbox{10mm}{36} & \parbox{10mm}{A} & \parbox{49mm}{Документ ''ОтчетПроизводстваЛГК''. Таблица ''Раскрои''. План. ГруппаРаскроев. Композиция. Код} & \parbox{11mm}{} & \parbox{89mm}{Код композиции в кодировке SYNCRO.} \\
\hline
\parbox[c][22mm]{25mm}{Reel's width} & \parbox{10mm}{55} & \parbox{10mm}{58} & \parbox{10mm}{4} & \parbox{10mm}{N} & \parbox{49mm}{Документ ''ОтчетПроизводстваЛГК''. Таблица ''Раскрои''. План. ГруппаРаскроев. Формат} & \parbox{11mm}{} & \parbox{89mm}{Ширина рулона, мм.} \\
\hline
\parbox[c][8mm]{25mm}{Type of trim} & \parbox{10mm}{59} & \parbox{10mm}{59} & \parbox{10mm}{1} & \parbox{10mm}{A} & \parbox{49mm}{} & \parbox{11mm}{} & \parbox{89mm}{Тип разреза на продольной резке. Значение ‘1’ (ASCII 31Hex) - с обрезью, ‘2’ (ASCII 32Hex) - без обрези.} \\
\hline
\parbox[c][13mm]{25mm}{Scoring gap distance} & \parbox{10mm}{60} & \parbox{10mm}{62} & \parbox{10mm}{3} & \parbox{10mm}{N} & \parbox{49mm}{Регистр ''Обмен с ГА: Глубина рилевок'' (Профиль, Марка, ТипРилевки)} & \parbox{11mm}{} & \parbox{89mm}{Расстояние между профилем рилевки верхнего вала по сравнению с профилем рилевки нижнего вала в 1/10mm.} \\
\hline
\parbox[c][20mm]{25mm}{Grammage} & \parbox{10mm}{63} & \parbox{10mm}{66} & \parbox{10mm}{4} & \parbox{10mm}{N} & \parbox{49mm}{Документ ''ОтчетПроизводстваЛГК''. Таблица ''Раскрои''. План. ГруппаРаскроев. Сумма(СлойN * ПлотностьСлояN)} & \parbox{11mm}{} & \parbox{89mm}{Общий граммаж всех слоёв раскроя.} \\
\hline
\parbox[c][20mm]{25mm}{Paper 1} & \parbox{10mm}{67} & \parbox{10mm}{78} & \parbox{10mm}{12} & \parbox{10mm}{A} & \parbox{49mm}{Документ ''ОтчетПроизводстваЛГК''. Таблица ''Раскрои''. План. ГруппаРаскроев. Слой1. Наименование} & \parbox{11mm}{Да} & \parbox{89mm}{Наименование слоя 1 бумаги.} \\
\hline
\parbox[c][20mm]{25mm}{Paper 2} & \parbox{10mm}{79} & \parbox{10mm}{90} & \parbox{10mm}{12} & \parbox{10mm}{A} & \parbox{49mm}{Документ ''ОтчетПроизводстваЛГК''. Таблица ''Раскрои''. План. ГруппаРаскроев. Слой2. Наименование} & \parbox{11mm}{Да} & \parbox{89mm}{Наименование слоя 2 бумаги.} \\
\hline
\parbox[c][20mm]{25mm}{Paper 3} & \parbox{10mm}{91} & \parbox{10mm}{102} & \parbox{10mm}{12} & \parbox{10mm}{A} & \parbox{49mm}{Документ ''ОтчетПроизводстваЛГК''. Таблица ''Раскрои''. План. ГруппаРаскроев. Слой3. Наименование} & \parbox{11mm}{Да} & \parbox{89mm}{Наименование слоя 3 бумаги.} \\
\hline
\parbox[c][20mm]{25mm}{Paper 4} & \parbox{10mm}{103} & \parbox{10mm}{114} & \parbox{10mm}{12} & \parbox{10mm}{A} & \parbox{49mm}{Документ ''ОтчетПроизводстваЛГК''. Таблица ''Раскрои''. План. ГруппаРаскроев. Слой4. Наименование} & \parbox{11mm}{Да} & \parbox{89mm}{Наименование слоя 4 бумаги.} \\
\hline
\parbox[c][20mm]{25mm}{Paper 5} & \parbox{10mm}{115} & \parbox{10mm}{126} & \parbox{10mm}{12} & \parbox{10mm}{A} & \parbox{49mm}{Документ ''ОтчетПроизводстваЛГК''. Таблица ''Раскрои''. План. ГруппаРаскроев. Слой5. Наименование} & \parbox{11mm}{Да} & \parbox{89mm}{Наименование слоя 5 бумаги.} \\
\hline
\parbox[c][20mm]{25mm}{Paper 6} & \parbox{10mm}{127} & \parbox{10mm}{138} & \parbox{10mm}{12} & \parbox{10mm}{A} & \parbox{49mm}{Документ ''ОтчетПроизводстваЛГК''. Таблица ''Раскрои''. План. ГруппаРаскроев. Слой6. Наименование} & \parbox{11mm}{Да} & \parbox{89mm}{Наименование слоя 6 бумаги.} \\
\hline
\parbox[c][20mm]{25mm}{Paper 7} & \parbox{10mm}{139} & \parbox{10mm}{150} & \parbox{10mm}{12} & \parbox{10mm}{A} & \parbox{49mm}{Документ ''ОтчетПроизводстваЛГК''. Таблица ''Раскрои''. План. ГруппаРаскроев. Слой7. Наименование} & \parbox{11mm}{Да} & \parbox{89mm}{Наименование слоя 7 бумаги.} \\
\hline
\parbox[c][20mm]{25mm}{Paper 1 Grammage} & \parbox{10mm}{151} & \parbox{10mm}{153} & \parbox{10mm}{3} & \parbox{10mm}{N} & \parbox{49mm}{Документ ''ОтчетПроизводстваЛГК''. Таблица ''Раскрои''. План. ГруппаРаскроев. Слой1. ПлотностьСлоя} & \parbox{11mm}{} & \parbox{89mm}{Граммаж бумаги слоя 1.} \\
\hline
\parbox[c][20mm]{25mm}{Paper 2 Grammage} & \parbox{10mm}{154} & \parbox{10mm}{156} & \parbox{10mm}{3} & \parbox{10mm}{N} & \parbox{49mm}{Документ ''ОтчетПроизводстваЛГК''. Таблица ''Раскрои''. План. ГруппаРаскроев. Слой2. ПлотностьСлоя} & \parbox{11mm}{} & \parbox{89mm}{Граммаж бумаги слоя 2.} \\
\hline
\parbox[c][20mm]{25mm}{Paper 3 Grammage} & \parbox{10mm}{157} & \parbox{10mm}{159} & \parbox{10mm}{3} & \parbox{10mm}{N} & \parbox{49mm}{Документ ''ОтчетПроизводстваЛГК''. Таблица ''Раскрои''. План. ГруппаРаскроев. Слой3. ПлотностьСлоя} & \parbox{11mm}{} & \parbox{89mm}{Граммаж бумаги слоя 3.} \\
\hline
\parbox[c][20mm]{25mm}{Paper 4 Grammage} & \parbox{10mm}{160} & \parbox{10mm}{162} & \parbox{10mm}{3} & \parbox{10mm}{N} & \parbox{49mm}{Документ ''ОтчетПроизводстваЛГК''. Таблица ''Раскрои''. План. ГруппаРаскроев. Слой4. ПлотностьСлоя} & \parbox{11mm}{} & \parbox{89mm}{Граммаж бумаги слоя 4.} \\
\hline
\parbox[c][20mm]{25mm}{Paper 5 Grammage} & \parbox{10mm}{163} & \parbox{10mm}{165} & \parbox{10mm}{3} & \parbox{10mm}{N} & \parbox{49mm}{Документ ''ОтчетПроизводстваЛГК''. Таблица ''Раскрои''. План. ГруппаРаскроев. Слой5. ПлотностьСлоя} & \parbox{11mm}{} & \parbox{89mm}{Граммаж бумаги слоя 5.} \\
\hline
\parbox[c][20mm]{25mm}{Paper 6 Grammage} & \parbox{10mm}{166} & \parbox{10mm}{168} & \parbox{10mm}{3} & \parbox{10mm}{N} & \parbox{49mm}{Документ ''ОтчетПроизводстваЛГК''. Таблица ''Раскрои''. План. ГруппаРаскроев. Слой6. ПлотностьСлоя} & \parbox{11mm}{} & \parbox{89mm}{Граммаж бумаги слоя 6.} \\
\hline
\parbox[c][20mm]{25mm}{Paper 7 Grammage} & \parbox{10mm}{169} & \parbox{10mm}{171} & \parbox{10mm}{3} & \parbox{10mm}{N} & \parbox{49mm}{Документ ''ОтчетПроизводстваЛГК''. Таблица ''Раскрои''. План. ГруппаРаскроев. Слой7. ПлотностьСлоя} & \parbox{11mm}{} & \parbox{89mm}{Граммаж бумаги слоя 7.} \\
\hline
{\bf \parbox[c][5mm]{25mm}{Заказ №1}} & \parbox{10mm}{} & \parbox{10mm}{} & \parbox{10mm}{} & \parbox{10mm}{} & \parbox{49mm}{} & \parbox{11mm}{} & \parbox{89mm}{} \\
\hline
\parbox[c][13mm]{25mm}{Order number} & \parbox{10mm}{172} & \parbox{10mm}{183} & \parbox{10mm}{12} & \parbox{10mm}{A} & \parbox{49mm}{Документ ''ОтчетПроизводстваЛГК''. Таблица ''Выработка'' Заказ. Номер} & \parbox{11mm}{} & \parbox{89mm}{Номер производственного заказа.} \\
\hline
\parbox[c][15mm]{25mm}{Customer} & \parbox{10mm}{184} & \parbox{10mm}{213} & \parbox{10mm}{30} & \parbox{10mm}{A} & \parbox{49mm}{Документ ''ОтчетПроизводстваЛГК''. Таблица ''Выработка''. Заказ. Контрагент. Наименование} & \parbox{11mm}{Да} & \parbox{89mm}{Наименование заказчика.} \\
\hline
\parbox[c][5mm]{25mm}{Destination} & \parbox{10mm}{214} & \parbox{10mm}{243} & \parbox{10mm}{30} & \parbox{10mm}{A} & \parbox{49mm}{Пустая строка} & \parbox{11mm}{} & \parbox{89mm}{Адрес доставки.} \\
\hline
\parbox[c][5mm]{25mm}{Customer's town} & \parbox{10mm}{244} & \parbox{10mm}{263} & \parbox{10mm}{20} & \parbox{10mm}{A} & \parbox{49mm}{Пустая строка} & \parbox{11mm}{} & \parbox{89mm}{Город доставки.} \\
\hline
\parbox[c][5mm]{25mm}{City code} & \parbox{10mm}{264} & \parbox{10mm}{265} & \parbox{10mm}{2} & \parbox{10mm}{A} & \parbox{49mm}{Пустая строка} & \parbox{11mm}{} & \parbox{89mm}{Код города доставки.} \\
\hline
\parbox[c][15mm]{25mm}{Customer's code} & \parbox{10mm}{266} & \parbox{10mm}{277} & \parbox{10mm}{12} & \parbox{10mm}{A} & \parbox{49mm}{Документ ''ОтчетПроизводстваЛГК''. Таблица ''Выработка''. Заказ. Контрагент. Код} & \parbox{11mm}{Да} & \parbox{89mm}{Код контрагента.} \\
\hline
\parbox[c][15mm]{25mm}{Level} & \parbox{10mm}{278} & \parbox{10mm}{278} & \parbox{10mm}{1} & \parbox{10mm}{A} & \parbox{49mm}{Документ ''ОтчетПроизводстваЛГК''. Таблица ''Выработка''. Стол} & \parbox{11mm}{} & \parbox{89mm}{Обозначение стола. Если номер стола 1, то ‘L’ (ASCII 4CHex), иначе ‘U’ (ASCII 55Hex).} \\
\hline
\parbox[c][15mm]{25mm}{Sheet length} & \parbox{10mm}{279} & \parbox{10mm}{282} & \parbox{10mm}{4} & \parbox{10mm}{N} & \parbox{49mm}{Документ ''ОтчетПроизводстваЛГК''. Таблица ''Выработка''. ДлинаЗаготовки} & \parbox{11mm}{} & \parbox{89mm}{Длина заготовки, мм.} \\
\hline
\parbox[c][20mm]{25mm}{Quantity} & \parbox{10mm}{283} & \parbox{10mm}{287} & \parbox{10mm}{5} & \parbox{10mm}{N} & \parbox{49mm}{Документ ''ОтчетПроизводстваЛГК''. Таблица ''Выработка''. КоличествоРезовПлан * КоличествоПолос} & \parbox{11mm}{} & \parbox{89mm}{Количество листов в заказе.} \\
\hline
\parbox[c][15mm]{25mm}{Outs} & \parbox{10mm}{288} & \parbox{10mm}{288} & \parbox{10mm}{1} & \parbox{10mm}{N} & \parbox{49mm}{Документ ''ОтчетПроизводстваЛГК''. Таблица ''Выработка''. КоличествоПолос} & \parbox{11mm}{} & \parbox{89mm}{Количество полос в заказе.} \\
\hline
\parbox[c][15mm]{25mm}{Sheet width} & \parbox{10mm}{289} & \parbox{10mm}{292} & \parbox{10mm}{4} & \parbox{10mm}{N} & \parbox{49mm}{Документ ''ОтчетПроизводстваЛГК''. Таблица ''Выработка''. ШиринаЗаготовки} & \parbox{11mm}{} & \parbox{89mm}{Ширина заготовки, мм.} \\
\hline
\parbox[c][18mm]{25mm}{Scorers' dimensions} & \parbox{10mm}{293} & \parbox{10mm}{391} & \parbox{10mm}{99} & \parbox{10mm}{N} & \parbox{49mm}{Документ ''ОтчетПроизводстваЛГК''. Таблица ''Раскрои''. План. СтруктураРаскроев. Рилевки} & \parbox{11mm}{} & \parbox{89mm}{Размеры рилевок на ящик, разделенные пробелом, мм.} \\
\hline
\parbox[c][16mm]{25mm}{Index of scorers’ group} & \parbox{10mm}{392} & \parbox{10mm}{392} & \parbox{10mm}{1} & \parbox{10mm}{A} & \parbox{49mm}{Регистр ''Обмен с ГА: Типы рилевок'' (ТипРилевки)} & \parbox{11mm}{} & \parbox{89mm}{Индекс группы рилевок. Значение по умолчанию - ‘B’ (ASCII 42Hex). Возможные значения:
‘A’ (ASCII 41Hex);
‘B’ (ASCII 42Hex);
‘C’ (ASCII 43Hex);
‘D’ (ASCII 44Hex).} \\
\hline
\parbox[c][22mm]{25mm}{Type of positioning} & \parbox{10mm}{393} & \parbox{10mm}{393} & \parbox{10mm}{1} & \parbox{10mm}{A} & \parbox{49mm}{Регистр ''Обмен с ГА: Типы рилевок'' (ТипРилевки)} & \parbox{11mm}{} & \parbox{89mm}{Тип позиционирования рилевок. Значение по умолчанию - ''По центру'', '-' (ASCII 2DHex). Возможные значения:
Нормальное смещение = '/' (ASCII 2FHex);
От точки к точке = 'X' (ASCII 58Hex);
Обратное смещение = '\' (ASCII 5CHex);
По центру = '-' (ASCII 2DHex).} \\
\hline
\parbox[c][20mm]{25mm}{Sheets per stack} & \parbox{10mm}{394} & \parbox{10mm}{397} & \parbox{10mm}{4} & \parbox{10mm}{N} & \parbox{49mm}{} & \parbox{11mm}{} & \parbox{89mm}{Количество листов в пачке. Значение по умолчанию - ‘0’ (ASCII 30Hex). Расчетный параметр, зависит от Типа изделия, Количества рядов пачек в паллете, Профиля, Размера заготовки и Максимального веса стопы (если указано в настройках СИСТЕМЫ).} \\
\hline
\parbox[c][15mm]{25mm}{Stacks per pallet} & \parbox{10mm}{398} & \parbox{10mm}{399} & \parbox{10mm}{2} & \parbox{10mm}{N} & \parbox{49mm}{Документ ''ОтчетПроизводстваЛГК''. Таблица ''Выработка''. КоличествоПолос} & \parbox{11mm}{} & \parbox{89mm}{Количество пачек на паллете. Равняется количеству полос.} \\
\hline
\parbox[c][20mm]{25mm}{Bundle/Pallet} & \parbox{10mm}{400} & \parbox{10mm}{400} & \parbox{10mm}{1} & \parbox{10mm}{N} & \parbox{49mm}{Документ ''ОтчетПроизводстваЛГК''. Таблица ''Выработка''. Заказ. ТехнологическаяКарта. ВариантУпаковки. Поддон} & \parbox{11mm}{} & \parbox{89mm}{Тип группировки: на паллете или связка. Значение по умолчанию - ‘0’ (ASCII 30Hex). Если указан признак ''Поддон'', то ‘0’ (ASCII 30Hex), иначе ‘1’ (ASCII 31Hex).} \\
\hline
\parbox[c][20mm]{25mm}{Take off side} & \parbox{10mm}{401} & \parbox{10mm}{401} & \parbox{10mm}{1} & \parbox{10mm}{N} & \parbox{49mm}{} & \parbox{11mm}{} & \parbox{89mm}{С какой стороны отводятся готовые заготовки: справа, слева или спереди. Значение по умолчанию - ‘0’ (ASCII 30Hex). Возможные значения:
Справа = ‘0’ (ASCII 30Hex);
Слева = ‘1’ (ASCII 31Hex);
Спереди = ‘2’ (ASCII 32Hex).} \\
\hline
\parbox[c][20mm]{25mm}{Sending of pallet} & \parbox{10mm}{402} & \parbox{10mm}{402} & \parbox{10mm}{1} & \parbox{10mm}{A} & \parbox{49mm}{} & \parbox{11mm}{} & \parbox{89mm}{Место перемещения паллет. Значение по умолчанию - ‘S’ (ASCII 53Hex). Возможные значения:
Переработка = ‘T’ (ASCII 54Hex);
Выпуск = ‘S’ (ASCII 53Hex);
Прочее = ‘A’ (ASCII 41Hex).} \\
\hline
\parbox[c][20mm]{25mm}{Material handling line} & \parbox{10mm}{403} & \parbox{10mm}{404} & \parbox{10mm}{2} & \parbox{10mm}{N} & \parbox{49mm}{Документ ''ОтчетПроизводстваЛГК''. Таблица ''Раскрои''. План. ЗаказыНаЛиниях. Оборудование. Код} & \parbox{11mm}{} & \parbox{89mm}{Номер линии, на которую будут поставлены заготовки, где шаг = 2. Значение по умолчанию - ‘0’ (ASCII 30Hex).} \\
\hline
\parbox[c][20mm]{25mm}{Name of the box factory machine} & \parbox{10mm}{405} & \parbox{10mm}{419} & \parbox{10mm}{15} & \parbox{10mm}{A} & \parbox{49mm}{Документ ''ОтчетПроизводстваЛГК''. Таблица ''Раскрои''. План. ЗаказыНаЛиниях. Оборудование. Наименование} & \parbox{11mm}{Да} & \parbox{89mm}{Наименование линии, на которую будут поставлены заготовки, где шаг = 2.} \\
\hline
\parbox[c][22mm]{25mm}{Balance} & \parbox{10mm}{420} & \parbox{10mm}{420} & \parbox{10mm}{1} & \parbox{10mm}{A} & \parbox{49mm}{} & \parbox{11mm}{} & \parbox{89mm}{Тип выработки заказа. Значение по умолчанию - ‘T’ (ASCII 54Hex). Возможные значения:
Полностью = ‘T’ (ASCII 54Hex);
Частично = ‘P’ (ASCII 50Hex);
Брак = ‘X’ (ASCII 58Hex).} \\
\hline
\parbox[c][20mm]{25mm}{Delivery date} & \parbox{10mm}{421} & \parbox{10mm}{428} & \parbox{10mm}{8} & \parbox{10mm}{A} & \parbox{49mm}{Документ ''ОтчетПроизводстваЛГК''. Таблица ''Выработка''. Заказ. ДатаОтгрузки} & \parbox{11mm}{} & \parbox{89mm}{Дата отгрузки заказа в формате (YYYYMMDD).} \\
\hline
\parbox[c][20mm]{25mm}{Product code} & \parbox{10mm}{429} & \parbox{10mm}{436} & \parbox{10mm}{8} & \parbox{10mm}{A} & \parbox{49mm}{Документ ''ОтчетПроизводстваЛГК''. Таблица ''Выработка''. Заказ. Номенклатура. Код} & \parbox{11mm}{Да} & \parbox{89mm}{Код продукции.} \\
\hline
\parbox[c][20mm]{25mm}{Data to print 1} & \parbox{10mm}{437} & \parbox{10mm}{472} & \parbox{10mm}{36} & \parbox{10mm}{A} & \parbox{49mm}{Документ ''ОтчетПроизводстваЛГК''. Таблица ''Раскрои''. План. СтруктураРаскроев. Композиция. КодДляFosber} & \parbox{11mm}{} & \parbox{89mm}{Код композиции в кодировке SYNCRO.} \\
\hline
\parbox[c][22mm]{25mm}{Data to print 2} & \parbox{10mm}{473} & \parbox{10mm}{487} & \parbox{10mm}{15} & \parbox{10mm}{A} & \parbox{49mm}{Документ ''ОтчетПроизводстваЛГК''. Таблица ''Выработка''. Заказ. ТехнологическаяКарта. ШиринаЗаготовки + 'x' + ДлинаЗаготовки} & \parbox{11mm}{} & \parbox{89mm}{Размеры для печати на этикетке.} \\
\hline
\parbox[c][5mm]{25mm}{Data to print 3} & \parbox{10mm}{488} & \parbox{10mm}{547} & \parbox{10mm}{60} & \parbox{10mm}{A} & \parbox{49mm}{Пустая строка} & \parbox{11mm}{} & \parbox{89mm}{Общие данные для печати на этикетке.} \\
\hline
\parbox[c][9mm]{25mm}{Data to print 4} & \parbox{10mm}{548} & \parbox{10mm}{557} & \parbox{10mm}{10} & \parbox{10mm}{N} & \parbox{49mm}{Пустая строка} & \parbox{11mm}{} & \parbox{89mm}{Данные по типу работы для печати.} \\*
\hline
\parbox[c][22mm]{25mm}{Pallet width} & \parbox{10mm}{558} & \parbox{10mm}{561} & \parbox{10mm}{4} & \parbox{10mm}{N} & \parbox{49mm}{Документ ''ОтчетПроизводстваЛГК''. Таблица ''Выработка''.Заказ. ТехнологическаяКарта. Поддон. ШиринаПоддона} & \parbox{11mm}{} & \parbox{89mm}{Ширина поддона.} \\
\hline
\parbox[c][22mm]{25mm}{Pallet length} & \parbox{10mm}{562} & \parbox{10mm}{565} & \parbox{10mm}{4} & \parbox{10mm}{N} & \parbox{49mm}{Документ ''ОтчетПроизводстваЛГК''. Таблица ''Выработка''.Заказ. ТехнологическаяКарта. Поддон. ДлинаПоддона} & \parbox{11mm}{} & \parbox{89mm}{Длина поддона.} \\
\hline
\parbox[c][8mm]{25mm}{Number of pallet per width} & \parbox{10mm}{566} & \parbox{10mm}{567} & \parbox{10mm}{2} & \parbox{10mm}{N} & \parbox{49mm}{} & \parbox{11mm}{} & \parbox{89mm}{Количество паллет по ширине. Значение по умолчанию - ‘0’ (ASCII 30Hex).} \\
\hline
\parbox[c][8mm]{25mm}{Number of pallet per length} & \parbox{10mm}{568} & \parbox{10mm}{569} & \parbox{10mm}{2} & \parbox{10mm}{N} & \parbox{49mm}{} & \parbox{11mm}{} & \parbox{89mm}{Количество паллет по длине. Значение по умолчанию - ‘0’ (ASCII 30Hex).} \\
\hline
\parbox[c][10mm]{25mm}{Double} & \parbox{10mm}{570} & \parbox{10mm}{570} & \parbox{10mm}{1} & \parbox{10mm}{N} & \parbox{49mm}{} & \parbox{11mm}{} & \parbox{89mm}{Двойная оптимизация. Значение по умолчанию - ‘0’ (ASCII 30Hex).
Возможные значения:
Нет = ‘0’ (ASCII 30Hex);
Да = ‘1’ (ASCII 31Hex).} \\
\hline
\parbox[c][22mm]{25mm}{Type of binding} & \parbox{10mm}{571} & \parbox{10mm}{571} & \parbox{10mm}{1} & \parbox{10mm}{N} & \parbox{49mm}{} & \parbox{11mm}{} & \parbox{89mm}{Тип обвязки. Значение по умолчанию - ‘0’ (ASCII 30Hex). Возможные значения: 
Нет = ‘0’ (ASCII 30Hex);
Лента = ‘1’ (ASCII 50Hex);
Лента + уголки = ‘X’ (ASCII 58Hex).} \\
\hline
\parbox[c][5mm]{25mm}{Binding code} & \parbox{10mm}{572} & \parbox{10mm}{572} & \parbox{10mm}{1} & \parbox{10mm}{N} & \parbox{49mm}{Пустая строка} & \parbox{11mm}{} & \parbox{89mm}{Код номенклатуры обвязки.} \\
\hline
\parbox[c][8mm]{25mm}{Number of edge-protections} & \parbox{10mm}{573} & \parbox{10mm}{574} & \parbox{10mm}{2} & \parbox{10mm}{N} & \parbox{49mm}{} & \parbox{11mm}{} & \parbox{89mm}{Количество защитных уголков. Значение по умолчанию - ‘0’ (ASCII 30Hex).} \\
\hline
\parbox[c][5mm]{25mm}{Notes} & \parbox{10mm}{575} & \parbox{10mm}{604} & \parbox{10mm}{30} & \parbox{10mm}{A} & \parbox{49mm}{Внутренний идентификатор задания СИСТЕМЫ.} & \parbox{11mm}{} & \parbox{89mm}{Данное поле содержит уникальный идентификатор раскроя, которое затем используется СИСТЕМОЙ для корректного позиционирования в документе для внесения изменений параметров раскроя или занесения факта выработки.} \\
\hline
{\bf \parbox[c][5mm]{25mm}{Заказ №2}} & \parbox{10mm}{} & \parbox{10mm}{} & \parbox{10mm}{} & \parbox{10mm}{} & \parbox{49mm}{} & \parbox{11mm}{} & \parbox{89mm}{} \\
\hline
\parbox[c][13mm]{25mm}{Order number} & \parbox{10mm}{605} & \parbox{10mm}{616} & \parbox{10mm}{12} & \parbox{10mm}{A} & \parbox{49mm}{Документ ''ОтчетПроизводстваЛГК''. Таблица ''Выработка'' Заказ. Номер} & \parbox{11mm}{} & \parbox{89mm}{Номер производственного заказа.} \\
\hline
\parbox[c][18mm]{25mm}{Customer} & \parbox{10mm}{617} & \parbox{10mm}{646} & \parbox{10mm}{30} & \parbox{10mm}{A} & \parbox{49mm}{Документ ''ОтчетПроизводстваЛГК''. Таблица ''Выработка''. Заказ. Контрагент. Наименование} & \parbox{11mm}{Да} & \parbox{89mm}{Наименование заказчика.} \\
\hline
\parbox[c][5mm]{25mm}{Destination} & \parbox{10mm}{647} & \parbox{10mm}{676} & \parbox{10mm}{30} & \parbox{10mm}{A} & \parbox{49mm}{Пустая строка} & \parbox{11mm}{} & \parbox{89mm}{Адрес доставки.} \\
\hline
\parbox[c][5mm]{25mm}{Customer's town} & \parbox{10mm}{677} & \parbox{10mm}{696} & \parbox{10mm}{20} & \parbox{10mm}{A} & \parbox{49mm}{Пустая строка} & \parbox{11mm}{} & \parbox{89mm}{Город доставки.} \\
\hline
\parbox[c][5mm]{25mm}{City code} & \parbox{10mm}{697} & \parbox{10mm}{698} & \parbox{10mm}{2} & \parbox{10mm}{A} & \parbox{49mm}{Пустая строка} & \parbox{11mm}{} & \parbox{89mm}{Код города доставки.} \\
\hline
\parbox[c][15mm]{25mm}{Customer's code} & \parbox{10mm}{699} & \parbox{10mm}{710} & \parbox{10mm}{12} & \parbox{10mm}{A} & \parbox{49mm}{Документ ''ОтчетПроизводстваЛГК''. Таблица ''Выработка''. Заказ. Контрагент. Код} & \parbox{11mm}{Да} & \parbox{89mm}{Код контрагента.} \\
\hline
\parbox[c][13mm]{25mm}{Level} & \parbox{10mm}{711} & \parbox{10mm}{711} & \parbox{10mm}{1} & \parbox{10mm}{A} & \parbox{49mm}{Документ ''ОтчетПроизводстваЛГК''. Таблица ''Выработка''. Стол} & \parbox{11mm}{} & \parbox{89mm}{Обозначение стола. Если номер стола 1, то ‘L’ (ASCII 4CHex), иначе ‘U’ (ASCII 55Hex).} \\
\hline
\parbox[c][15mm]{25mm}{Sheet length} & \parbox{10mm}{712} & \parbox{10mm}{715} & \parbox{10mm}{4} & \parbox{10mm}{N} & \parbox{49mm}{Документ ''ОтчетПроизводстваЛГК''. Таблица ''Выработка''. ДлинаЗаготовки} & \parbox{11mm}{} & \parbox{89mm}{Длина заготовки, мм.} \\
\hline
\parbox[c][18mm]{25mm}{Quantity} & \parbox{10mm}{716} & \parbox{10mm}{720} & \parbox{10mm}{5} & \parbox{10mm}{N} & \parbox{49mm}{Документ ''ОтчетПроизводстваЛГК''. Таблица ''Выработка''. КоличествоРезовПлан * КоличествоПолос} & \parbox{11mm}{} & \parbox{89mm}{Количество листов в заказе.} \\
\hline
\parbox[c][15mm]{25mm}{Outs} & \parbox{10mm}{721} & \parbox{10mm}{721} & \parbox{10mm}{1} & \parbox{10mm}{N} & \parbox{49mm}{Документ ''ОтчетПроизводстваЛГК''. Таблица ''Выработка''. КоличествоПолос} & \parbox{11mm}{} & \parbox{89mm}{Количество полос в заказе.} \\
\hline
\parbox[c][20mm]{25mm}{Sheet width} & \parbox{10mm}{722} & \parbox{10mm}{725} & \parbox{10mm}{4} & \parbox{10mm}{N} & \parbox{49mm}{Документ ''ОтчетПроизводстваЛГК''. Таблица ''Выработка''. ШиринаЗаготовки} & \parbox{11mm}{} & \parbox{89mm}{Ширина заготовки, мм.} \\
\hline
\parbox[c][20mm]{25mm}{Scorers' dimensions} & \parbox{10mm}{726} & \parbox{10mm}{824} & \parbox{10mm}{99} & \parbox{10mm}{N} & \parbox{49mm}{Документ ''ОтчетПроизводстваЛГК''. Таблица ''Раскрои''. План. СтруктураРаскроев. Рилевки} & \parbox{11mm}{} & \parbox{89mm}{Размеры рилевок на ящик, разделенные пробелом, мм.} \\
\hline
\parbox[c][22mm]{25mm}{Index of scorers’ group} & \parbox{10mm}{825} & \parbox{10mm}{825} & \parbox{10mm}{1} & \parbox{10mm}{A} & \parbox{49mm}{Регистр ''Обмен с ГА: Типы рилевок'' (ТипРилевки)} & \parbox{11mm}{} & \parbox{89mm}{Индекс группы рилевок. Значение по умолчанию - ‘B’ (ASCII 42Hex). Возможные значения:
‘A’ (ASCII 41Hex);
‘B’ (ASCII 42Hex);
‘C’ (ASCII 43Hex);
‘D’ (ASCII 44Hex).} \\
\hline
\parbox[c][20mm]{25mm}{Type of positioning} & \parbox{10mm}{826} & \parbox{10mm}{826} & \parbox{10mm}{1} & \parbox{10mm}{A} & \parbox{49mm}{Регистр ''Обмен с ГА: Типы рилевок'' (ТипРилевки)} & \parbox{11mm}{} & \parbox{89mm}{Тип позиционирования рилевок. Значение по умолчанию - по центру, '-' (ASCII 2DHex). Возможные значения:
Нормальное смещение = '/' (ASCII 2FHex);
От точки к точке = 'X' (ASCII 58Hex);
Обратное смещение = '\' (ASCII 5CHex);
По центру = '-' (ASCII 2DHex).}  \\
\hline
\parbox[c][22mm]{25mm}{Sheets per stack} & \parbox{10mm}{827} & \parbox{10mm}{830} & \parbox{10mm}{4} & \parbox{10mm}{N} & \parbox{49mm}{} & \parbox{11mm}{} & \parbox{89mm}{Количество листов в пачке. Значение по умолчанию - ‘0’ (ASCII 30Hex). Расчетный параметр, зависит от Типа изделия, Количества рядов пачек в паллете, Профиля, Размера заготовки и Максимального веса стопы (если указано в настройках СИСТЕМЫ).} \\
\hline
\parbox[c][22mm]{25mm}{Stacks per pallet} & \parbox{10mm}{831} & \parbox{10mm}{832} & \parbox{10mm}{2} & \parbox{10mm}{N} & \parbox{49mm}{Документ ''ОтчетПроизводстваЛГК''. Таблица ''Выработка''. КоличествоПолос} & \parbox{11mm}{} & \parbox{89mm}{Количество пачек на паллете. Равняется количеству полос.} \\
\hline
\parbox[c][22mm]{25mm}{Bundle/Pallet} & \parbox{10mm}{833} & \parbox{10mm}{833} & \parbox{10mm}{1} & \parbox{10mm}{N} & \parbox{49mm}{Документ ''ОтчетПроизводстваЛГК''. Таблица ''Выработка''. Заказ. ТехнологическаяКарта. ВариантУпаковки. Поддон} & \parbox{11mm}{} & \parbox{89mm}{Тип группировки: на паллете или связка. Значение по умолчанию - ‘0’ (ASCII 30Hex). Если указан признак ''Поддон'', то ‘0’ (ASCII 30Hex), иначе ‘1’ (ASCII 31Hex).} \\
\hline
\parbox[c][26mm]{25mm}{Take off side} & \parbox{10mm}{834} & \parbox{10mm}{834} & \parbox{10mm}{1} & \parbox{10mm}{N} & \parbox{49mm}{} & \parbox{11mm}{} & \parbox{89mm}{С какой стороны отводятся готовые заготовки: справа, слева или спереди. Значение по умолчанию - ‘0’ (ASCII 30Hex). Возможные значения:
Справа = ‘0’ (ASCII 30Hex);
Слева = ‘1’ (ASCII 31Hex);
Спереди = ‘2’ (ASCII 32Hex).} \\
\hline
\parbox[c][22mm]{25mm}{Sending of pallet} & \parbox{10mm}{835} & \parbox{10mm}{835} & \parbox{10mm}{1} & \parbox{10mm}{A} & \parbox{49mm}{} & \parbox{11mm}{} & \parbox{89mm}{Место перемещения паллет. Значение по умолчанию - ‘S’ (ASCII 53Hex). Возможные значения:
Переработка = ‘T’ (ASCII 54Hex);
Выпуск = ‘S’ (ASCII 53Hex);
Прочее = ‘A’ (ASCII 41Hex).} \\
\hline
\parbox[c][20mm]{25mm}{Material handling line} & \parbox{10mm}{836} & \parbox{10mm}{837} & \parbox{10mm}{2} & \parbox{10mm}{N} & \parbox{49mm}{Документ ''ОтчетПроизводстваЛГК''. Таблица ''Раскрои''. План. ЗаказыНаЛиниях. Оборудование. Код} & \parbox{11mm}{} & \parbox{89mm}{Номер линии, на которую будут поставлены заготовки, где шаг = 2. Значение по умолчанию - ‘0’ (ASCII 30Hex).} \\
\hline
\parbox[c][20mm]{25mm}{Name of the box factory machine} & \parbox{10mm}{838} & \parbox{10mm}{852} & \parbox{10mm}{15} & \parbox{10mm}{A} & \parbox{49mm}{Документ ''ОтчетПроизводстваЛГК''. Таблица ''Раскрои''. План. ЗаказыНаЛиниях. Оборудование. Наименование} & \parbox{11mm}{Да} & \parbox{89mm}{Наименование линии, на которую будут поставлены заготовки, где шаг = 2.} \\
\hline
\parbox[c][22mm]{25mm}{Balance} & \parbox{10mm}{853} & \parbox{10mm}{853} & \parbox{10mm}{1} & \parbox{10mm}{A} & \parbox{49mm}{} & \parbox{11mm}{} & \parbox{89mm}{Тип выработки заказа. Значение по умолчанию - ‘T’ (ASCII 54Hex). Возможные значения:
Полностью = ‘T’ (ASCII 54Hex);
Частично = ‘P’ (ASCII 50Hex);
Брак = ‘X’ (ASCII 58Hex).} \\
\hline
\parbox[c][20mm]{25mm}{Delivery date} & \parbox{10mm}{854} & \parbox{10mm}{861} & \parbox{10mm}{8} & \parbox{10mm}{A} & \parbox{49mm}{Документ ''ОтчетПроизводстваЛГК''. Таблица ''Выработка''. Заказ. ДатаОтгрузки} & \parbox{11mm}{} & \parbox{89mm}{Дата отгрузки заказа в формате (YYYYMMDD).} \\
\hline
\parbox[c][20mm]{25mm}{Product code} & \parbox{10mm}{862} & \parbox{10mm}{869} & \parbox{10mm}{8} & \parbox{10mm}{A} & \parbox{49mm}{Документ ''ОтчетПроизводстваЛГК''. Таблица ''Выработка''. Заказ. Номенклатура. Код} & \parbox{11mm}{Да} & \parbox{89mm}{Код продукции.} \\
\hline
\parbox[c][20mm]{25mm}{Data to print 1} & \parbox{10mm}{870} & \parbox{10mm}{905} & \parbox{10mm}{36} & \parbox{10mm}{A} & \parbox{49mm}{Документ ''ОтчетПроизводстваЛГК''. Таблица ''Раскрои''. План. СтруктураРаскроев. Композиция. КодДляFosber} & \parbox{11mm}{} & \parbox{89mm}{Код композиции в кодировке SYNCRO.} \\
\hline
\parbox[c][22mm]{25mm}{Data to print 2} & \parbox{10mm}{906} & \parbox{10mm}{920} & \parbox{10mm}{15} & \parbox{10mm}{A} & \parbox{49mm}{Документ ''ОтчетПроизводстваЛГК''. Таблица ''Выработка''. Заказ. ТехнологическаяКарта. ШиринаЗаготовки + 'x' + ДлинаЗаготовки} & \parbox{11mm}{} & \parbox{89mm}{Размеры для печати на этикетке.} \\
\hline
\parbox[c][5mm]{25mm}{Data to print 3} & \parbox{10mm}{921} & \parbox{10mm}{980} & \parbox{10mm}{60} & \parbox{10mm}{A} & \parbox{49mm}{Пустая строка} & \parbox{11mm}{} & \parbox{89mm}{Общие данные для печати на этикетке.} \\
\hline
\parbox[c][9mm]{25mm}{Data to print 4} & \parbox{10mm}{981} & \parbox{10mm}{990} & \parbox{10mm}{10} & \parbox{10mm}{N} & \parbox{49mm}{Пустая строка} & \parbox{11mm}{} & \parbox{89mm}{Данные по типу работы для печати.} \\
\hline
\parbox[c][22mm]{25mm}{Pallet width} & \parbox{10mm}{991} & \parbox{10mm}{994} & \parbox{10mm}{4} & \parbox{10mm}{N} & \parbox{49mm}{Документ ''ОтчетПроизводстваЛГК''. Таблица ''Выработка''.Заказ. ТехнологическаяКарта. Поддон. ШиринаПоддона} & \parbox{11mm}{} & \parbox{89mm}{Ширина поддона.} \\
\hline
\parbox[c][20mm]{25mm}{Pallet length} & \parbox{10mm}{995} & \parbox{10mm}{998} & \parbox{10mm}{4} & \parbox{10mm}{N} & \parbox{49mm}{Документ ''ОтчетПроизводстваЛГК''. Таблица ''Выработка''.Заказ. ТехнологическаяКарта. Поддон. ДлинаПоддона} & \parbox{11mm}{} & \parbox{89mm}{Длина поддона.} \\
\hline
\parbox[c][13mm]{25mm}{Number of pallet per width} & \parbox{10mm}{999} & \parbox{10mm}{1000} & \parbox{10mm}{2} & \parbox{10mm}{N} & \parbox{49mm}{} & \parbox{11mm}{} & \parbox{89mm}{Количество паллет по ширине. Значение по умолчанию - ‘0’ (ASCII 30Hex).} \\
\hline
\parbox[c][13mm]{25mm}{Number of pallet per length} & \parbox{10mm}{1001} & \parbox{10mm}{1002} & \parbox{10mm}{2} & \parbox{10mm}{N} & \parbox{49mm}{} & \parbox{11mm}{} & \parbox{89mm}{Количество паллет по длине. Значение по умолчанию - ‘0’ (ASCII 30Hex).} \\
\hline
\parbox[c][15mm]{25mm}{Double} & \parbox{10mm}{1003} & \parbox{10mm}{1003} & \parbox{10mm}{1} & \parbox{10mm}{N} & \parbox{49mm}{} & \parbox{11mm}{} & \parbox{89mm}{Двойная оптимизация. Значение по умолчанию - ‘0’ (ASCII 30Hex). Возможные значения:
Нет = ‘0’ (ASCII 30Hex);
Да = ‘1’ (ASCII 31Hex).}\\
\hline
\parbox[c][18mm]{25mm}{Type of binding} & \parbox{10mm}{1004} & \parbox{10mm}{1004} & \parbox{10mm}{1} & \parbox{10mm}{N} & \parbox{49mm}{} & \parbox{11mm}{} & \parbox{89mm}{Тип обвязки. Значение по умолчанию - ‘0’ (ASCII 30Hex). Возможные значения: 
Нет = ‘0’ (ASCII 30Hex);
Лента = ‘1’ (ASCII 50Hex);
Лента + уголки = ‘X’ (ASCII 58Hex).} \\
\hline
\parbox[c][5mm]{25mm}{Binding code} & \parbox{10mm}{1005} & \parbox{10mm}{1005} & \parbox{10mm}{1} & \parbox{10mm}{N} & \parbox{49mm}{Пустая строка} & \parbox{11mm}{} & \parbox{89mm}{Код номенклатуры обвязки.} \\
\hline
\parbox[c][8mm]{25mm}{Number of edge-protections} & \parbox{10mm}{1006} & \parbox{10mm}{1007} & \parbox{10mm}{2} & \parbox{10mm}{N} & \parbox{49mm}{} & \parbox{11mm}{} & \parbox{89mm}{Количество защитных уголков. Значение по умолчанию - ‘0’ (ASCII 30Hex).} \\
\hline
\parbox[c][20mm]{25mm}{Notes} & \parbox{10mm}{1008} & \parbox{10mm}{1037} & \parbox{10mm}{30} & \parbox{10mm}{A} & \parbox{49mm}{Внутренний идентификатор задания СИСТЕМЫ.} & \parbox{11mm}{} & \parbox{89mm}{Данное поле содержит уникальный идентификатор раскроя, которое затем используется СИСТЕМОЙ для корректного позиционирования в документе для внесения изменений параметров раскроя или занесения факта выработки.} \\
\hline
{\bf \parbox[c][5mm]{25mm}{Заказ №3}} & \parbox{10mm}{} & \parbox{10mm}{} & \parbox{10mm}{} & \parbox{10mm}{} & \parbox{49mm}{} & \parbox{11mm}{} & \parbox{89mm}{} \\
\hline
\parbox[c][13mm]{25mm}{Order number} & \parbox{10mm}{1038} & \parbox{10mm}{1049} & \parbox{10mm}{12} & \parbox{10mm}{N} & \parbox{49mm}{Документ ''ОтчетПроизводстваЛГК''. Таблица ''Выработка'' Заказ. Номер} & \parbox{11mm}{} & \parbox{89mm}{Номер производственного заказа.} \\
\hline
\parbox[c][18mm]{25mm}{Customer} & \parbox{10mm}{1050} & \parbox{10mm}{1079} & \parbox{10mm}{30} & \parbox{10mm}{N} & \parbox{49mm}{Документ ''ОтчетПроизводстваЛГК''. Таблица ''Выработка''. Заказ. Контрагент. Наименование} & \parbox{11mm}{Да} & \parbox{89mm}{Наименование заказчика.} \\
\hline
\parbox[c][5mm]{25mm}{Destination} & \parbox{10mm}{1080} & \parbox{10mm}{1109} & \parbox{10mm}{30} & \parbox{10mm}{A} & \parbox{49mm}{Пустая строка} & \parbox{11mm}{} & \parbox{89mm}{Адрес доставки.} \\
\hline
\parbox[c][5mm]{25mm}{Customer's town} & \parbox{10mm}{1110} & \parbox{10mm}{1129} & \parbox{10mm}{20} & \parbox{10mm}{A} & \parbox{49mm}{Пустая строка} & \parbox{11mm}{} & \parbox{89mm}{Город доставки.} \\
\hline
\parbox[c][5mm]{25mm}{City code} & \parbox{10mm}{1130} & \parbox{10mm}{1131} & \parbox{10mm}{2} & \parbox{10mm}{A} & \parbox{49mm}{Пустая строка} & \parbox{11mm}{} & \parbox{89mm}{Код города доставки.} \\
\hline
\parbox[c][18mm]{25mm}{Customer's code} & \parbox{10mm}{1132} & \parbox{10mm}{1143} & \parbox{10mm}{12} & \parbox{10mm}{A} & \parbox{49mm}{Документ ''ОтчетПроизводстваЛГК''. Таблица ''Выработка''. Заказ. Контрагент. Код} & \parbox{11mm}{Да} & \parbox{89mm}{Код контрагента.} \\
\hline
\parbox[c][15mm]{25mm}{Level} & \parbox{10mm}{1144} & \parbox{10mm}{1144} & \parbox{10mm}{1} & \parbox{10mm}{N} & \parbox{49mm}{Документ ''ОтчетПроизводстваЛГК''. Таблица ''Выработка''. Стол} & \parbox{11mm}{} & \parbox{89mm}{Обозначение стола. Если номер стола 1, то ‘L’ (ASCII 4CHex), иначе ‘U’ (ASCII 55Hex).} \\
\hline
\parbox[c][18mm]{25mm}{Sheet length} & \parbox{10mm}{1145} & \parbox{10mm}{1148} & \parbox{10mm}{4} & \parbox{10mm}{N} & \parbox{49mm}{Документ ''ОтчетПроизводстваЛГК''. Таблица ''Выработка''. ДлинаЗаготовки} & \parbox{11mm}{} & \parbox{89mm}{Длина заготовки, мм.} \\
\hline
\parbox[c][18mm]{25mm}{Quantity} & \parbox{10mm}{1149} & \parbox{10mm}{1153} & \parbox{10mm}{5} & \parbox{10mm}{N} & \parbox{49mm}{Документ ''ОтчетПроизводстваЛГК''. Таблица ''Выработка''. КоличествоРезовПлан * КоличествоПолос} & \parbox{11mm}{} & \parbox{89mm}{Количество листов в заказе.} \\
\hline
\parbox[c][18mm]{25mm}{Outs} & \parbox{10mm}{1154} & \parbox{10mm}{1154} & \parbox{10mm}{1} & \parbox{10mm}{N} & \parbox{49mm}{Документ ''ОтчетПроизводстваЛГК''. Таблица ''Выработка''. КоличествоПолос} & \parbox{11mm}{} & \parbox{89mm}{Количество полос в заказе.} \\
\hline
\parbox[c][18mm]{25mm}{Sheet width} & \parbox{10mm}{1155} & \parbox{10mm}{1158} & \parbox{10mm}{4} & \parbox{10mm}{N} & \parbox{49mm}{Документ ''ОтчетПроизводстваЛГК''. Таблица ''Выработка''. ШиринаЗаготовки} & \parbox{11mm}{} & \parbox{89mm}{Ширина заготовки, мм.} \\
\hline
\parbox[c][18mm]{25mm}{Scorers' dimensions} & \parbox{10mm}{1159} & \parbox{10mm}{1257} & \parbox{10mm}{99} & \parbox{10mm}{N} & \parbox{49mm}{Документ ''ОтчетПроизводстваЛГК''. Таблица ''Раскрои''. План. СтруктураРаскроев. Рилевки} & \parbox{11mm}{} & \parbox{89mm}{Размеры рилевок на ящик, разделенные пробелом, мм.} \\
\hline
\parbox[c][22mm]{25mm}{Index of scorers’ group} & \parbox{10mm}{1258} & \parbox{10mm}{1258} & \parbox{10mm}{1} & \parbox{10mm}{A} & \parbox{49mm}{Регистр ''Обмен с ГА: Типы рилевок'' (ТипРилевки)} & \parbox{11mm}{} & \parbox{89mm}{Индекс группы рилевок. Значение по умолчанию - ‘B’ (ASCII 42Hex). Возможные значения:
‘A’ (ASCII 41Hex);
‘B’ (ASCII 42Hex);
‘C’ (ASCII 43Hex);
‘D’ (ASCII 44Hex).} \\
\hline
\parbox[c][20mm]{25mm}{Type of positioning} & \parbox{10mm}{1259} & \parbox{10mm}{1259} & \parbox{10mm}{1} & \parbox{10mm}{A} & \parbox{49mm}{Регистр ''Обмен с ГА: Типы рилевок'' (ТипРилевки)} & \parbox{11mm}{} & \parbox{89mm}{Тип позиционирования рилевок. Значение по умолчанию - по центру, '-' (ASCII 2DHex). Возможные значения:
Нормальное смещение = '/' (ASCII 2FHex);
От точки к точке = 'X' (ASCII 58Hex);
Обратное смещение = '\' (ASCII 5CHex);
По центру = '-' (ASCII 2DHex).}  \\
\hline
\parbox[c][20mm]{25mm}{Sheets per stack} & \parbox{10mm}{1260} & \parbox{10mm}{1263} & \parbox{10mm}{4} & \parbox{10mm}{N} & \parbox{49mm}{} & \parbox{11mm}{} & \parbox{89mm}{Количество листов в пачке. Значение по умолчанию - ‘0’ (ASCII 30Hex). Расчетный параметр, зависит от Типа изделия, Количества рядов пачек в паллете, Профиля, Размера заготовки и Максимального веса стопы (если указано в настройках СИСТЕМЫ).} \\
\hline
\parbox[c][20mm]{25mm}{Stacks per pallet} & \parbox{10mm}{1264} & \parbox{10mm}{1265} & \parbox{10mm}{2} & \parbox{10mm}{N} & \parbox{49mm}{Документ ''ОтчетПроизводстваЛГК''. Таблица ''Выработка''. КоличествоПолос} & \parbox{11mm}{} & \parbox{89mm}{Количество пачек на паллете. Равняется количеству полос.} \\
\hline
\parbox[c][22mm]{25mm}{Bundle/Pallet} & \parbox{10mm}{1266} & \parbox{10mm}{1266} & \parbox{10mm}{1} & \parbox{10mm}{N} & \parbox{49mm}{Документ ''ОтчетПроизводстваЛГК''. Таблица ''Выработка''. Заказ. ТехнологическаяКарта. ВариантУпаковки. Поддон} & \parbox{11mm}{} & \parbox{89mm}{Тип группировки: на паллете или связка. Значение по умолчанию - ‘0’ (ASCII 30Hex). Если указан признак ''Поддон'', то ‘0’ (ASCII 30Hex), иначе ‘1’ (ASCII 31Hex).} \\
\hline
\parbox[c][26mm]{25mm}{Take off side} & \parbox{10mm}{1267} & \parbox{10mm}{1267} & \parbox{10mm}{1} & \parbox{10mm}{N} & \parbox{49mm}{} & \parbox{11mm}{} & \parbox{89mm}{С какой стороны отводятся готовые заготовки: справа, слева или спереди. Значение по умолчанию - ‘0’ (ASCII 30Hex). Возможные значения:
Справа = ‘0’ (ASCII 30Hex);
Слева = ‘1’ (ASCII 31Hex);
Спереди = ‘2’ (ASCII 32Hex).} \\
\hline
\parbox[c][22mm]{25mm}{Sending of pallet} & \parbox{10mm}{1268} & \parbox{10mm}{1268} & \parbox{10mm}{1} & \parbox{10mm}{A} & \parbox{49mm}{} & \parbox{11mm}{} & \parbox{89mm}{Место перемещения паллет. Значение по умолчанию - ‘S’ (ASCII 53Hex). Возможные значения:
Переработка = ‘T’ (ASCII 54Hex);
Выпуск = ‘S’ (ASCII 53Hex);
Прочее = ‘A’ (ASCII 41Hex).} \\
\hline
\parbox[c][18mm]{25mm}{Material handling line} & \parbox{10mm}{1269} & \parbox{10mm}{1270} & \parbox{10mm}{2} & \parbox{10mm}{N} & \parbox{49mm}{Документ ''ОтчетПроизводстваЛГК''. Таблица ''Раскрои''. План. ЗаказыНаЛиниях. Оборудование. Код} & \parbox{11mm}{} & \parbox{89mm}{Номер линии, на которую будут поставлены заготовки, где шаг = 2. Значение по умолчанию - ‘0’ (ASCII 30Hex).} \\
\hline
\parbox[c][18mm]{25mm}{Name of the box factory machine} & \parbox{10mm}{1271} & \parbox{10mm}{1285} & \parbox{10mm}{15} & \parbox{10mm}{A} & \parbox{49mm}{Документ ''ОтчетПроизводстваЛГК''. Таблица ''Раскрои''. План. ЗаказыНаЛиниях. Оборудование. Наименование} & \parbox{11mm}{Да} & \parbox{89mm}{Наименование линии, на которую будут поставлены заготовки, где шаг = 2.} \\
\hline
\parbox[c][22mm]{25mm}{Balance} & \parbox{10mm}{1286} & \parbox{10mm}{1286} & \parbox{10mm}{1} & \parbox{10mm}{A} & \parbox{49mm}{} & \parbox{11mm}{} & \parbox{89mm}{Тип выработки заказа. Значение по умолчанию - ‘T’ (ASCII 54Hex). Возможные значения:
Полностью = ‘T’ (ASCII 54Hex);
Частично = ‘P’ (ASCII 50Hex);
Брак = ‘X’ (ASCII 58Hex).} \\
\hline
\parbox[c][18mm]{25mm}{Delivery date} & \parbox{10mm}{1287} & \parbox{10mm}{1294} & \parbox{10mm}{8} & \parbox{10mm}{A} & \parbox{49mm}{Документ ''ОтчетПроизводстваЛГК''. Таблица ''Выработка''. Заказ. ДатаОтгрузки} & \parbox{11mm}{} & \parbox{89mm}{Дата отгрузки заказа в формате (YYYYMMDD).} \\
\hline
\parbox[c][18mm]{25mm}{Product code} & \parbox{10mm}{1295} & \parbox{10mm}{1302} & \parbox{10mm}{8} & \parbox{10mm}{A} & \parbox{49mm}{Документ ''ОтчетПроизводстваЛГК''. Таблица ''Выработка''. Заказ. Номенклатура. Код} & \parbox{11mm}{Да} & \parbox{89mm}{Код продукции.} \\
\hline
\parbox[c][18mm]{25mm}{Data to print 1} & \parbox{10mm}{1303} & \parbox{10mm}{1338} & \parbox{10mm}{36} & \parbox{10mm}{A} & \parbox{49mm}{Документ ''ОтчетПроизводстваЛГК''. Таблица ''Раскрои''. План. СтруктураРаскроев. Композиция. КодДляFosber} & \parbox{11mm}{} & \parbox{89mm}{Код композиции в кодировке SYNCRO.} \\
\hline
\parbox[c][22mm]{25mm}{Data to print 2} & \parbox{10mm}{1339} & \parbox{10mm}{1353} & \parbox{10mm}{15} & \parbox{10mm}{A} & \parbox{49mm}{Документ ''ОтчетПроизводстваЛГК''. Таблица ''Выработка''. Заказ. ТехнологическаяКарта. ШиринаЗаготовки + 'x' + ДлинаЗаготовки} & \parbox{11mm}{} & \parbox{89mm}{Размеры для печати на этикетке.} \\
\hline
\parbox[c][5mm]{25mm}{Data to print 3} & \parbox{10mm}{1354} & \parbox{10mm}{1413} & \parbox{10mm}{60} & \parbox{10mm}{A} & \parbox{49mm}{Пустая строка} & \parbox{11mm}{} & \parbox{89mm}{Общие данные для печати на этикетке.} \\
\hline
\parbox[c][9mm]{25mm}{Data to print 4} & \parbox{10mm}{1414} & \parbox{10mm}{1423} & \parbox{10mm}{10} & \parbox{10mm}{N} & \parbox{49mm}{Пустая строка} & \parbox{11mm}{} & \parbox{89mm}{Данные по типу работы для печати.} \\
\hline
\parbox[c][22mm]{25mm}{Pallet width} & \parbox{10mm}{1424} & \parbox{10mm}{1427} & \parbox{10mm}{4} & \parbox{10mm}{N} & \parbox{49mm}{Документ ''ОтчетПроизводстваЛГК''. Таблица ''Выработка''.Заказ. ТехнологическаяКарта. Поддон. ШиринаПоддона} & \parbox{11mm}{} & \parbox{89mm}{Ширина поддона.} \\
\hline
\parbox[c][22mm]{25mm}{Pallet length} & \parbox{10mm}{1428} & \parbox{10mm}{1431} & \parbox{10mm}{4} & \parbox{10mm}{N} & \parbox{49mm}{Документ ''ОтчетПроизводстваЛГК''. Таблица ''Выработка''.Заказ. ТехнологическаяКарта. Поддон. ДлинаПоддона} & \parbox{11mm}{} & \parbox{89mm}{Длина поддона.} \\
\hline
\parbox[c][8mm]{25mm}{Number of pallet per width} & \parbox{10mm}{1432} & \parbox{10mm}{1433} & \parbox{10mm}{2} & \parbox{10mm}{N} & \parbox{49mm}{} & \parbox{11mm}{} & \parbox{89mm}{Количество паллет по ширине. Значение по умолчанию - ‘0’ (ASCII 30Hex).} \\
\hline
\parbox[c][8mm]{25mm}{Number of pallet per length} & \parbox{10mm}{1434} & \parbox{10mm}{1435} & \parbox{10mm}{2} & \parbox{10mm}{N} & \parbox{49mm}{} & \parbox{11mm}{} & \parbox{89mm}{Количество паллет по длине. Значение по умолчанию - ‘0’ (ASCII 30Hex).} \\
\hline
\parbox[c][10mm]{25mm}{Double} & \parbox{10mm}{1436} & \parbox{10mm}{1436} & \parbox{10mm}{1} & \parbox{10mm}{N} & \parbox{49mm}{} & \parbox{11mm}{} & \parbox{89mm}{Двойная оптимизация.
Возможные значения:
Нет = ‘0’ (ASCII 30Hex);
Да = ‘1’ (ASCII 31Hex).} \\
\hline
\parbox[c][22mm]{25mm}{Type of binding} & \parbox{10mm}{1437} & \parbox{10mm}{1437} & \parbox{10mm}{1} & \parbox{10mm}{N} & \parbox{49mm}{} & \parbox{11mm}{} & \parbox{89mm}{Тип обвязки. Значение по умолчанию - ‘0’ (ASCII 30Hex). Возможные значения: 
Нет = ‘0’ (ASCII 30Hex);
Лента = ‘1’ (ASCII 50Hex);
Лента + уголки = ‘X’ (ASCII 58Hex).} \\
\hline
\parbox[c][5mm]{25mm}{Binding code} & \parbox{10mm}{1438} & \parbox{10mm}{1438} & \parbox{10mm}{1} & \parbox{10mm}{N} & \parbox{49mm}{Пустая строка} & \parbox{11mm}{} & \parbox{89mm}{Код номенклатуры обвязки.} \\
\hline
\parbox[c][8mm]{25mm}{Number of edge-protections} & \parbox{10mm}{1439} & \parbox{10mm}{1440} & \parbox{10mm}{2} & \parbox{10mm}{N} & \parbox{49mm}{} & \parbox{11mm}{} & \parbox{89mm}{Количество защитных уголков. Значение по умолчанию - ‘0’ (ASCII 30Hex).} \\
\hline
\parbox[c][20mm]{25mm}{Notes} & \parbox{10mm}{1441} & \parbox{10mm}{1470} & \parbox{10mm}{30} & \parbox{10mm}{A} & \parbox{49mm}{Внутренний идентификатор задания СИСТЕМЫ.} & \parbox{11mm}{} & \parbox{89mm}{Данное поле содержит уникальный идентификатор раскроя, которое затем используется СИСТЕМОЙ для корректного позиционирования в документе для внесения изменений параметров раскроя или занесения факта выработки.} \\
\hline
\caption{Структура блока параметров раскроя для передачи из СИСТЕМЫ в SYNCRO, а также для получения информации обратно.}\label{tab:fosber_block}
\end{longtable}

\end{landscape} 

\normalsize


% ТЕКУЩИЙ СТАТУС

\scriptsize
\begin{landscape}
Структура блока сообщения текущего статуса из SYNCRO в СИСТЕМУ представлена в таблице \ref{tab:fosber_status}.
% Table generated by Excel2LaTeX from sheet 'Обратно'
\scriptsize

\begin{longtable}{|p{45mm}|p{6mm}|p{6mm}|p{8mm}|p{6mm}|p{70mm}|p{80mm}|}
\hline
\parbox[c][10mm]{45mm}{\centering Поле} & \parbox{6mm}{\centering С} & \parbox{6mm}{\centering По} & \parbox{8mm}{\centering Длина} & \parbox{5mm}{\centering Тип} & \parbox{70mm}{\centering Комментарий}  & \parbox{80mm}{\centering Opti-Corrugated} \\
\hline
\parbox[c][12mm]{45mm}{\raggedright Number of orders in the waiting job list} & \parbox[c]{9mm}{\centering 1} & \parbox[c]{9mm}{\centering 3} & \parbox[c]{11mm}{\centering 3} & \parbox[c]{10mm}{\centering N} & \parbox[c]{80mm}{\raggedright Количество заказов в очереди.} & \parbox[c]{80mm}{\raggedright} \\
\hline
\parbox[c][10mm]{45mm}{Actual speed} & \parbox{9mm}{4} & \parbox{9mm}{7} & \parbox{11mm}{4} & \parbox{10mm}{N} & \parbox{70mm}{Текущая скорость гофроагрегата.} & \parbox{80mm}{Документ ''ОтчетПроизводстваЛГК''. Форма. Текстовое поле ''Инфо Fosber''.} \\
\hline
\parbox[c][10mm]{45mm}{Average speed} & \parbox{9mm}{8} & \parbox{9mm}{11} & \parbox{11mm}{4} & \parbox{10mm}{N} & \parbox{70mm}{Средняя скорость гофроагрегата.} & \parbox{80mm}{Документ ''ОтчетПроизводстваЛГК''. Форма. Текстовое поле ''Инфо Fosber''.} \\
\hline
\parbox[c][13mm]{45mm}{Linear Meters produced from the beginning of the order} & \parbox{9mm}{12} & \parbox{9mm}{20} & \parbox{11mm}{9} & \parbox{10mm}{N} & \parbox{70mm}{Количество метров, произведенных по заказу.} & \parbox{50mm}{} \\
\hline
\parbox[c][10mm]{45mm}{Current Run ID 1} & \parbox{9mm}{21} & \parbox{9mm}{24} & \parbox{11mm}{4} & \parbox{10mm}{N} & \parbox{70mm}{Текущий идентификатор раскроя (1000 - 9999).} & \parbox{80mm}{Документ ''ОтчетПроизводстваЛГК''. Форма. Текстовое поле ''Инфо Fosber''.} \\
\hline
\parbox[c][10mm]{45mm}{Next Run ID 1} & \parbox{9mm}{25} & \parbox{9mm}{28} & \parbox{11mm}{4} & \parbox{10mm}{N} & \parbox{70mm}{Следующий идентификатор раскроя (1000 - 9999).} & \parbox{80mm}{Документ ''ОтчетПроизводстваЛГК''. Форма. Текстовое поле ''Инфо Fosber''.} \\
\hline
\parbox[c][5mm]{45mm}{Total good meters} & \parbox{9mm}{29} & \parbox{9mm}{37} & \parbox{11mm}{9} & \parbox{10mm}{N} & \parbox{70mm}{Всего метров, произведенных с начала смены.} & \parbox{80mm}{} \\
\hline
\parbox[c][12mm]{45mm}{Total scrap meters from shift's start} & \parbox{9mm}{38} & \parbox{9mm}{46} & \parbox{11mm}{9} & \parbox{10mm}{N} & \parbox{70mm}{Всего брака с начала смены, в метрах.} & \parbox{80mm}{} \\
\hline
\parbox[c][5mm]{45mm}{Free} & \parbox{9mm}{47} & \parbox{9mm}{62} & \parbox{11mm}{16} & \parbox{10mm}{N} & \parbox{70mm}{Не используется.} & \parbox{80mm}{} \\
\hline
\parbox[c][5mm]{45mm}{Rest meters} & \parbox{9mm}{63} & \parbox{9mm}{71} & \parbox{11mm}{9} & \parbox{10mm}{N} & \parbox{70mm}{Остаток метров до смены бумаги.} & \parbox{80mm}{} \\
\hline
{\bf\parbox[c][13mm]{45mm}{Стол №1}} & \parbox{9mm}{} & \parbox{9mm}{} & \parbox{11mm}{} & \parbox{10mm}{} & \parbox{80mm}{} & \parbox{70mm}{} \\
\hline
\parbox[c][10mm]{45mm}{Run ID 1} & \parbox{9mm}{72} & \parbox{9mm}{75} & \parbox{11mm}{4} & \parbox{10mm}{N} & \parbox{70mm}{Номер связки.} & \parbox{80mm}{Документ ''ОтчетПроизводстваЛГК''. Таблица ''Раскрои''. Номер связки} \\
\hline
\parbox[c][10mm]{45mm}{Order number} & \parbox{9mm}{76} & \parbox{9mm}{87} & \parbox{11mm}{12} & \parbox{10mm}{N} & \parbox{70mm}{Номер заказа.} & \parbox{80mm}{Документ ''ОтчетПроизводстваЛГК''. Таблица ''Выработка''. Заказ. Номер} \\
\hline
\parbox[c][8mm]{45mm}{Nr of discharges} & \parbox{9mm}{88} & \parbox{9mm}{89} & \parbox{11mm}{2} & \parbox{10mm}{N} & \parbox{70mm}{Количество смен рулонов с начала заказа.} & \parbox{80mm}{} \\
\hline
\parbox[c][10mm]{45mm}{Out} & \parbox{9mm}{90} & \parbox{9mm}{91} & \parbox{11mm}{2} & \parbox{10mm}{N} & \parbox{70mm}{Количество полос в заказе.} & \parbox{80mm}{Документ ''ОтчетПроизводстваЛГК''. Таблица ''Выработка''. КоличествоПолос} \\
\hline
\parbox[c][10mm]{45mm}{Number sheets per stack} & \parbox{9mm}{92} & \parbox{9mm}{95} & \parbox{11mm}{4} & \parbox{10mm}{N} & \parbox{70mm}{Количество листов в пачке.} & \parbox{80mm}{Документ ''ОтчетПроизводстваЛГК''. Таблица ''Выработка''. КоличествоПолос} \\
\hline
\parbox[c][5mm]{45mm}{Aoc} & \parbox{9mm}{96} & \parbox{9mm}{96} & \parbox{11mm}{1} & \parbox{10mm}{N} & \parbox{80mm}{Флаг последней пачки.} & \parbox{80mm}{} \\
\hline
{\bf \parbox[c][5mm]{45mm}{Стол  №2}} & \parbox{9mm}{} & \parbox{9mm}{} & \parbox{11mm}{} & \parbox{10mm}{} & \parbox{70mm}{} & \parbox{80mm}{} \\
\hline
\parbox[c][10mm]{45mm}{Run ID 1} & \parbox{9mm}{97} & \parbox{9mm}{100} & \parbox{11mm}{4} & \parbox{10mm}{N} & \parbox{70mm}{Номер связки.} & \parbox{80mm}{Документ ''ОтчетПроизводстваЛГК''. Таблица ''Раскрои''. НомерСвязки} \\
\hline
\parbox[c][10mm]{45mm}{Order number} & \parbox{9mm}{101} & \parbox{9mm}{112} & \parbox{11mm}{12} & \parbox{10mm}{N} & \parbox{70mm}{Номер заказа.} & \parbox{80mm}{Документ ''ОтчетПроизводстваЛГК''. Таблица ''Выработка''. Заказ. Номер} \\
\hline
\parbox[c][8mm]{45mm}{Nr of discharges} & \parbox{9mm}{113} & \parbox{9mm}{114} & \parbox{11mm}{2} & \parbox{10mm}{N} & \parbox{70mm}{Количество смен рулонов с начала заказа.} & \parbox{80mm}{} \\
\hline
\parbox[c][10mm]{45mm}{Out} & \parbox{9mm}{115} & \parbox{9mm}{116} & \parbox{11mm}{2} & \parbox{10mm}{N} & \parbox{70mm}{Количество полос в заказе.} & \parbox{80mm}{Документ ''ОтчетПроизводстваЛГК''. Таблица ''Выработка''. КоличествоПолос} \\
\hline
\parbox[c][5mm]{45mm}{Number sheets per stack} & \parbox{9mm}{117} & \parbox{9mm}{120} & \parbox{11mm}{4} & \parbox{10mm}{N} & \parbox{70mm}{Количество листов в пачке.} & \parbox{80mm}{} \\
\hline
\parbox[c][5mm]{45mm}{Aoc} & \parbox{9mm}{121} & \parbox{9mm}{121} & \parbox{11mm}{1} & \parbox{10mm}{N} & \parbox{70mm}{Флаг последней пачки.} & \parbox{80mm}{} \\
\hline
{\bf \parbox[c][5mm]{45mm}{Стол №3}} & \parbox{9mm}{} & \parbox{9mm}{} & \parbox{11mm}{} & \parbox{10mm}{} & \parbox{70mm}{} & \parbox{80mm}{} \\
\hline
\parbox[c][10mm]{45mm}{Run ID 1} & \parbox{9mm}{122} & \parbox{9mm}{125} & \parbox{11mm}{4} & \parbox{10mm}{N} & \parbox{70mm}{Номер связки.} & \parbox{80mm}{} \\
\hline
\parbox[c][10mm]{45mm}{Order number} & \parbox{9mm}{126} & \parbox{9mm}{137} & \parbox{11mm}{12} & \parbox{10mm}{N} & \parbox{70mm}{Номер заказа.} & \parbox{80mm}{} \\
\hline
\parbox[c][8mm]{45mm}{Nr of discharges} & \parbox{9mm}{138} & \parbox{9mm}{139} & \parbox{11mm}{2} & \parbox{10mm}{N} & \parbox{70mm}{Количество смен рулонов с начала заказа.} & \parbox{80mm}{} \\
\hline
\parbox[c][10mm]{45mm}{Out} & \parbox{9mm}{140} & \parbox{9mm}{141} & \parbox{11mm}{2} & \parbox{10mm}{N} & \parbox{70mm}{Количество полос.} & \parbox{80mm}{Документ ''ОтчетПроизводстваЛГК''. Таблица ''Выработка''. КоличествоПолос} \\
\hline
\parbox[c][5mm]{45mm}{Number sheets per stack} & \parbox{9mm}{142} & \parbox{9mm}{145} & \parbox{11mm}{4} & \parbox{10mm}{N} & \parbox{70mm}{Количество листов в пачке.} & \parbox{80mm}{} \\
\hline
\parbox[c][5mm]{45mm}{Aoc} & \parbox{9mm}{146} & \parbox{9mm}{146} & \parbox{11mm}{1} & \parbox{10mm}{N} & \parbox{70mm}{Флаг последней пачки.} & \parbox{80mm}{} \\
\hline
\caption{Структура блока сообщения текущего статуса из SYNCRO в СИСТЕМУ}\label{tab:fosber_status}
\end{longtable}  
\normalsize

\end{landscape} 
\normalsize





% СПИСОК ЗАДАНИЙ

\scriptsize
\begin{landscape}
Структура блока сообщения о номерах выполненных и невыполненных заданий из SYNCRO в СИСТЕМУ представлена в таблице \ref{tab:fosber_list}.
\scriptsize

\begin{longtable}{|p{45mm}|p{6mm}|p{6mm}|p{8mm}|p{6mm}|p{70mm}|p{80mm}|}
\hline
\parbox[c][10mm]{45mm}{\centering Поле} & \parbox{6mm}{\centering С} & \parbox{6mm}{\centering По} & \parbox{8mm}{\centering Длина} & \parbox{5mm}{\centering Тип} & \parbox{70mm}{\centering Комментарий}  & \parbox{80mm}{\centering Opti-Corrugated} \\
\hline
\parbox[c][5mm]{45mm}{Number of orders in the list} & \parbox{9mm}{1} & \parbox{9mm}{3} & \parbox{11mm}{3} & \parbox{10mm}{N} & \parbox{70mm}{Количество раскроев в списке.} & \parbox{80mm}{} \\
\hline
\parbox[c][10mm]{45mm}{Run ID 1 (1) + Mod ID (1)} & \parbox{9mm}{4} & \parbox{9mm}{9} & \parbox{11mm}{6} & \parbox{10mm}{N} & \parbox{70mm}{Номер связки раскроя №1 из списка.} & \parbox{80mm}{По данному идентификатору определяется либо успех загрузки раскроя SYNCRO, либо его готовность.} \\
\hline
\parbox[c][5mm]{45mm}{Run ID 1 (2) + Mod ID (2)} & \parbox{9mm}{4} & \parbox{9mm}{9} & \parbox{11mm}{6} & \parbox{10mm}{N} & \parbox{70mm}{Номер связки раскроя №2 из списка.} & \parbox{80mm}{} \\
\hline
\parbox[c][5mm]{45mm}{...} & \parbox{9mm}{} & \parbox{9mm}{} & \parbox{11mm}{} & \parbox{10mm}{} & \parbox{70mm}{} & \parbox{80mm}{} \\
\hline
\parbox[c][5mm]{45mm}{Run ID 1 (300) + Mod ID (300)} & \parbox{9mm}{1798} & \parbox{9mm}{1803} & \parbox{11mm}{6} & \parbox{10mm}{N} & \parbox{70mm}{Номер связки раскроя №300 из списка.} & \parbox{80mm}{} \\
\hline
\caption{Структура блока сообщения о номерах выполненных и невыполненных заданий из SYNCRO в СИСТЕМУ}\label{tab:fosber_list}
\end{longtable}  
\normalsize

\end{landscape} 
\normalsize




% ДАННЫЕ ВЫРАБОТКИ

\scriptsize
\begin{landscape}
Структура блока сообщения о выполненных раскроях из SYNCRO в СИСТЕМУ представлена в таблице \ref{tab:fosber_completed}.
\scriptsize

\begin{longtable}{|p{45mm}|p{6mm}|p{6mm}|p{8mm}|p{6mm}|p{70mm}|p{80mm}|}
\hline
\parbox[c][10mm]{45mm}{\centering Поле} & \parbox{6mm}{\centering С} & \parbox{6mm}{\centering По} & \parbox{8mm}{\centering Длина} & \parbox{5mm}{\centering Тип} & \parbox{70mm}{\centering Комментарий}  & \parbox{80mm}{\centering Opti-Corrugated} \\
\hline
\parbox[c][10mm]{45mm}{Run ID 1} & \parbox{9mm}{1} & \parbox{9mm}{4} & \parbox{11mm}{4} & \parbox{10mm}{N} & \parbox{70mm}{Номер связки.} & \parbox{80mm}{Документ ''ОтчетПроизводстваЛГК''. Таблица ''Раскрои''. Номер связки} \\
\hline
\parbox[c][10mm]{45mm}{Modifications Id} & \parbox{9mm}{5} & \parbox{9mm}{6} & \parbox{11mm}{2} & \parbox{10mm}{N} & \parbox{70mm}{Номер модификации.} & \parbox{80mm}{} \\
\hline
\parbox[c][10mm]{45mm}{Run ID 2} & \parbox{9mm}{7} & \parbox{9mm}{17} & \parbox{11mm}{11} & \parbox{10mm}{A} & \parbox{70mm}{Идентификатор задания.} & \parbox{80mm}{Документ ''ОтчетПроизводстваЛГК''. Таблица ''Раскрои''. ID Задания} \\
\hline
\parbox[c][10mm]{45mm}{Shift manager} & \parbox{9mm}{18} & \parbox{9mm}{22} & \parbox{11mm}{5} & \parbox{10mm}{A} & \parbox{70mm}{Мастер смены.} & \parbox{80mm}{} \\
\hline
\parbox[c][10mm]{45mm}{Start Date} & \parbox{9mm}{23} & \parbox{9mm}{30} & \parbox{11mm}{8} & \parbox{10mm}{N} & \parbox{70mm}{Дата начала.} & \parbox{80mm}{Документ ''ОтчетПроизводстваЛГК''. Таблица ''Раскрои''. ВремяНачалаОбработкиРаскроя} \\
\hline
\parbox[c][10mm]{45mm}{End Date} & \parbox{9mm}{31} & \parbox{9mm}{38} & \parbox{11mm}{8} & \parbox{10mm}{N} & \parbox{70mm}{Дата окончания.} & \parbox{80mm}{Документ ''ОтчетПроизводстваЛГК''. Таблица ''Раскрои''. ВремяОкончанияОбработкиРаскроя} \\
\hline
\parbox[c][10mm]{45mm}{Start Time} & \parbox{9mm}{39} & \parbox{9mm}{46} & \parbox{11mm}{8} & \parbox{10mm}{A} & \parbox{70mm}{Время начала.} & \parbox{80mm}{Документ ''ОтчетПроизводстваЛГК''. Таблица ''Раскрои''. ВремяНачалаОбработкиРаскроя} \\
\hline
\parbox[c][10mm]{45mm}{End Time} & \parbox{9mm}{47} & \parbox{9mm}{54} & \parbox{11mm}{8} & \parbox{10mm}{N} & \parbox{70mm}{Время окончания.} & \parbox{80mm}{Документ ''ОтчетПроизводстваЛГК''. Таблица ''Раскрои''. ВремяОкончанияОбработкиРаскроя} \\
\hline
\parbox[c][10mm]{45mm}{Shift ID} & \parbox{9mm}{55} & \parbox{9mm}{55} & \parbox{11mm}{1} & \parbox{10mm}{A} & \parbox{70mm}{Идентификатор смены.} & \parbox{80mm}{} \\
\hline
\parbox[c][10mm]{45mm}{Number of people per shift} & \parbox{9mm}{56} & \parbox{9mm}{57} & \parbox{11mm}{2} & \parbox{10mm}{N} & \parbox{70mm}{Количество работников смены.} & \parbox{80mm}{} \\
\hline
\parbox[c][10mm]{45mm}{Stop time} & \parbox{9mm}{58} & \parbox{9mm}{65} & \parbox{11mm}{8} & \parbox{10mm}{A} & \parbox{70mm}{Время простоев в течение задания.} & \parbox{80mm}{} \\
\hline
\parbox[c][10mm]{45mm}{Number of stops} & \parbox{9mm}{66} & \parbox{9mm}{70} & \parbox{11mm}{5} & \parbox{10mm}{A} & \parbox{70mm}{Количество остановов в течение задания.} & \parbox{80mm}{} \\
\hline
\parbox[c][10mm]{45mm}{Worked time} & \parbox{9mm}{71} & \parbox{9mm}{78} & \parbox{11mm}{8} & \parbox{10mm}{A} & \parbox{70mm}{Полезное время работы.} & \parbox{80mm}{} \\
\hline
\parbox[c][10mm]{45mm}{Sheets for upper stack} & \parbox{9mm}{79} & \parbox{9mm}{83} & \parbox{11mm}{5} & \parbox{10mm}{N} & \parbox{70mm}{Количество листов верхнего стола.} & \parbox{80mm}{} \\
\hline
\parbox[c][10mm]{45mm}{Sheets for medium stack} & \parbox{9mm}{84} & \parbox{9mm}{88} & \parbox{11mm}{5} & \parbox{10mm}{N} & \parbox{70mm}{Количество листов центрального стола.} & \parbox{80mm}{} \\
\hline
\parbox[c][10mm]{45mm}{Sheets for lower stack} & \parbox{9mm}{89} & \parbox{9mm}{93} & \parbox{11mm}{5} & \parbox{10mm}{N} & \parbox{70mm}{Количество листов нижнего стола.} & \parbox{80mm}{} \\
\hline
\parbox[c][10mm]{45mm}{Board width} & \parbox{9mm}{94} & \parbox{9mm}{97} & \parbox{11mm}{4} & \parbox{10mm}{N} & \parbox{70mm}{Формат.} & \parbox{80mm}{} \\
\hline
\parbox[c][10mm]{45mm}{Actual produced meters} & \parbox{9mm}{98} & \parbox{9mm}{106} & \parbox{11mm}{9} & \parbox{10mm}{N} & \parbox{70mm}{Выработано всего, метров.} & \parbox{80mm}{} \\
\hline
\parbox[c][10mm]{45mm}{Theoretical meters to be produced} & \parbox{9mm}{107} & \parbox{9mm}{111} & \parbox{11mm}{5} & \parbox{10mm}{N} & \parbox{70mm}{Плановая выработка, метров.} & \parbox{80mm}{} \\
\hline
\parbox[c][10mm]{45mm}{Flute type} & \parbox{9mm}{112} & \parbox{9mm}{114} & \parbox{11mm}{3} & \parbox{10mm}{A} & \parbox{70mm}{Профиль.} & \parbox{80mm}{} \\
\hline
\parbox[c][10mm]{45mm}{Flute composition code} & \parbox{9mm}{115} & \parbox{9mm}{150} & \parbox{11mm}{36} & \parbox{10mm}{A} & \parbox{70mm}{Код композиции.} & \parbox{80mm}{} \\
\hline
\parbox[c][10mm]{45mm}{Average production speed} & \parbox{9mm}{151} & \parbox{9mm}{154} & \parbox{11mm}{4} & \parbox{10mm}{N} & \parbox{70mm}{Средняя скорость выработки.} & \parbox{80mm}{} \\
\hline
\parbox[c][10mm]{45mm}{Run ID 1} & \parbox{9mm}{155} & \parbox{9mm}{158} & \parbox{11mm}{4} & \parbox{10mm}{N} & \parbox{70mm}{Номер связки.} & \parbox{80mm}{} \\
\hline
{\bf \parbox[c][10mm]{45mm}{Заказ №1}} & \parbox{9mm}{} & \parbox{9mm}{} & \parbox{11mm}{} & \parbox{10mm}{} & \parbox{70mm}{} & \parbox{80mm}{} \\
\hline
\parbox[c][10mm]{45mm}{Order number} & \parbox{9mm}{159} & \parbox{9mm}{170} & \parbox{11mm}{12} & \parbox{10mm}{A} & \parbox{70mm}{Номер заказа.} & \parbox{80mm}{Документ ''ОтчетПроизводстваЛГК''. Таблица ''Выработка''. Заказ. Номер} \\
\hline
\parbox[c][10mm]{45mm}{Box width} & \parbox{9mm}{171} & \parbox{9mm}{174} & \parbox{11mm}{4} & \parbox{10mm}{N} & \parbox{70mm}{Ширина заготовки.} & \parbox{80mm}{Документ ''ОтчетПроизводстваЛГК''. Таблица ''Выработка''. ШиринаЗаготовки} \\
\hline
\parbox[c][10mm]{45mm}{Box length} & \parbox{9mm}{175} & \parbox{9mm}{178} & \parbox{11mm}{4} & \parbox{10mm}{N} & \parbox{70mm}{Длина заготовки.} & \parbox{80mm}{Документ ''ОтчетПроизводстваЛГК''. Таблица ''Выработка''. ДлинаЗаготовки} \\
\hline
\parbox[c][10mm]{45mm}{Scorers’ measures} & \parbox{9mm}{179} & \parbox{9mm}{277} & \parbox{11mm}{99} & \parbox{10mm}{N} & \parbox{70mm}{Рилевки.} & \parbox{80mm}{} \\
\hline
\parbox[c][10mm]{45mm}{Outs} & \parbox{9mm}{278} & \parbox{9mm}{278} & \parbox{11mm}{1} & \parbox{10mm}{N} & \parbox{70mm}{Количество полос.} & \parbox{80mm}{Документ ''ОтчетПроизводстваЛГК''. Таблица ''Выработка''. КоличествоПолос} \\
\hline
\parbox[c][10mm]{45mm}{Sheets carried out} & \parbox{9mm}{279} & \parbox{9mm}{284} & \parbox{11mm}{6} & \parbox{10mm}{N} & \parbox{70mm}{Количество заготовок.} & \parbox{80mm}{Документ ''ОтчетПроизводстваЛГК''. Таблица ''Выработка''. КоличествоЗаготовокФакт} \\
\hline
\parbox[c][10mm]{45mm}{Scrap sheets} & \parbox{9mm}{285} & \parbox{9mm}{290} & \parbox{11mm}{6} & \parbox{10mm}{N} & \parbox{70mm}{Брак заготовок.} & \parbox{80mm}{Документ ''ОтчетПроизводстваЛГК''. Таблица ''Выработка''. БракВРаботе} \\
\hline
\parbox[c][10mm]{45mm}{Balance} & \parbox{9mm}{291} & \parbox{9mm}{291} & \parbox{11mm}{1} & \parbox{10mm}{A} & \parbox{70mm}{Тип выработки заказа.} & \parbox{80mm}{} \\
\hline
\parbox[c][10mm]{45mm}{Sending of pallet} & \parbox{9mm}{292} & \parbox{9mm}{292} & \parbox{11mm}{1} & \parbox{10mm}{A} & \parbox{70mm}{Тип выпуска заказа.} & \parbox{80mm}{} \\
\hline
\parbox[c][10mm]{45mm}{Material handling line} & \parbox{9mm}{293} & \parbox{9mm}{294} & \parbox{11mm}{2} & \parbox{10mm}{A} & \parbox{70mm}{Код следующей линии.} & \parbox{80mm}{} \\
\hline
\parbox[c][10mm]{45mm}{Name of the box factory machine} & \parbox{9mm}{295} & \parbox{9mm}{309} & \parbox{11mm}{15} & \parbox{10mm}{A} & \parbox{70mm}{Наименование следующей линии.} & \parbox{80mm}{} \\
\hline
{\bf \parbox[c][10mm]{45mm}{Заказ №2}} & \parbox{9mm}{} & \parbox{9mm}{} & \parbox{11mm}{} & \parbox{10mm}{} & \parbox{70mm}{} & \parbox{80mm}{} \\
\hline
\parbox[c][10mm]{45mm}{Order number} & \parbox{9mm}{310} & \parbox{9mm}{321} & \parbox{11mm}{12} & \parbox{10mm}{A} & \parbox{70mm}{Номер заказа.} & \parbox{80mm}{Документ ''ОтчетПроизводстваЛГК''. Таблица ''Выработка''. Заказ. Номер} \\
\hline
\parbox[c][10mm]{45mm}{Box width} & \parbox{9mm}{322} & \parbox{9mm}{325} & \parbox{11mm}{4} & \parbox{10mm}{N} & \parbox{70mm}{Ширина заготовки.} & \parbox{80mm}{Документ ''ОтчетПроизводстваЛГК''. Таблица ''Выработка''. ШиринаЗаготовки} \\
\hline
\parbox[c][10mm]{45mm}{Box length} & \parbox{9mm}{326} & \parbox{9mm}{329} & \parbox{11mm}{4} & \parbox{10mm}{N} & \parbox{70mm}{Длина заготовки.} & \parbox{80mm}{Документ ''ОтчетПроизводстваЛГК''. Таблица ''Выработка''. ДлинаЗаготовки} \\
\hline
\parbox[c][10mm]{45mm}{Scorers’ measures} & \parbox{9mm}{330} & \parbox{9mm}{428} & \parbox{11mm}{99} & \parbox{10mm}{N} & \parbox{70mm}{Рилевки.} & \parbox{80mm}{} \\
\hline
\parbox[c][10mm]{45mm}{Outs} & \parbox{9mm}{429} & \parbox{9mm}{429} & \parbox{11mm}{1} & \parbox{10mm}{N} & \parbox{70mm}{Количество полос.} & \parbox{80mm}{Документ ''ОтчетПроизводстваЛГК''. Таблица ''Выработка''. КоличествоПолос} \\
\hline
\parbox[c][10mm]{45mm}{Sheets carried out} & \parbox{9mm}{430} & \parbox{9mm}{435} & \parbox{11mm}{6} & \parbox{10mm}{N} & \parbox{70mm}{Количество заготовок.} & \parbox{80mm}{Документ ''ОтчетПроизводстваЛГК''. Таблица ''Выработка''. КоличествоЗаготовокФакт} \\
\hline
\parbox[c][10mm]{45mm}{Scrap sheets} & \parbox{9mm}{436} & \parbox{9mm}{441} & \parbox{11mm}{6} & \parbox{10mm}{N} & \parbox{70mm}{Брак заготовок.} & \parbox{80mm}{Документ ''ОтчетПроизводстваЛГК''. Таблица ''Выработка''. БракВРаботе} \\
\hline
\parbox[c][10mm]{45mm}{Balance} & \parbox{9mm}{442} & \parbox{9mm}{442} & \parbox{11mm}{1} & \parbox{10mm}{A} & \parbox{70mm}{Тип выработки заказа.} & \parbox{80mm}{} \\
\hline
\parbox[c][10mm]{45mm}{Sending of pallet} & \parbox{9mm}{443} & \parbox{9mm}{443} & \parbox{11mm}{1} & \parbox{10mm}{A} & \parbox{70mm}{Тип выпуска заказа.} & \parbox{80mm}{} \\
\hline
\parbox[c][10mm]{45mm}{Material handling line} & \parbox{9mm}{444} & \parbox{9mm}{445} & \parbox{11mm}{2} & \parbox{10mm}{A} & \parbox{70mm}{Код следующей линии.} & \parbox{80mm}{} \\
\hline
\parbox[c][10mm]{45mm}{Name of the box factory machine} & \parbox{9mm}{446} & \parbox{9mm}{460} & \parbox{11mm}{15} & \parbox{10mm}{A} & \parbox{70mm}{Наименование следующей линии.} & \parbox{80mm}{} \\
\hline
{\bf \parbox[c][10mm]{45mm}{Заказ №3}} & \parbox{9mm}{} & \parbox{9mm}{} & \parbox{11mm}{} & \parbox{10mm}{} & \parbox{70mm}{} & \parbox{80mm}{} \\
\hline
\parbox[c][10mm]{45mm}{Order number} & \parbox{9mm}{461} & \parbox{9mm}{472} & \parbox{11mm}{12} & \parbox{10mm}{A} & \parbox{70mm}{Номер заказа.} & \parbox{80mm}{Документ ''ОтчетПроизводстваЛГК''. Таблица ''Выработка''. Заказ. Номер} \\
\hline
\parbox[c][10mm]{45mm}{Box width} & \parbox{9mm}{473} & \parbox{9mm}{476} & \parbox{11mm}{4} & \parbox{10mm}{N} & \parbox{70mm}{Ширина заготовки.} & \parbox{80mm}{Документ ''ОтчетПроизводстваЛГК''. Таблица ''Выработка''. ШиринаЗаготовки} \\
\hline
\parbox[c][10mm]{45mm}{Box length} & \parbox{9mm}{477} & \parbox{9mm}{480} & \parbox{11mm}{4} & \parbox{10mm}{N} & \parbox{70mm}{Длина заготовки.} & \parbox{80mm}{Документ ''ОтчетПроизводстваЛГК''. Таблица ''Выработка''. ДлинаЗаготовки} \\
\hline
\parbox[c][10mm]{45mm}{Scorers’ measures} & \parbox{9mm}{481} & \parbox{9mm}{579} & \parbox{11mm}{99} & \parbox{10mm}{N} & \parbox{70mm}{Рилевки.} & \parbox{80mm}{} \\
\hline
\parbox[c][10mm]{45mm}{Outs} & \parbox{9mm}{580} & \parbox{9mm}{580} & \parbox{11mm}{1} & \parbox{10mm}{N} & \parbox{70mm}{Количество полос.} & \parbox{80mm}{Документ ''ОтчетПроизводстваЛГК''. Таблица ''Выработка''. КоличествоПолос} \\
\hline
\parbox[c][10mm]{45mm}{Sheets carried out} & \parbox{9mm}{581} & \parbox{9mm}{586} & \parbox{11mm}{6} & \parbox{10mm}{N} & \parbox{70mm}{Количество заготовок.} & \parbox{80mm}{Документ ''ОтчетПроизводстваЛГК''. Таблица ''Выработка''. КоличествоЗаготовокФакт} \\
\hline
\parbox[c][10mm]{45mm}{Scrap sheets} & \parbox{9mm}{587} & \parbox{9mm}{592} & \parbox{11mm}{6} & \parbox{10mm}{N} & \parbox{70mm}{Брак заготовок.} & \parbox{80mm}{Документ ''ОтчетПроизводстваЛГК''. Таблица ''Выработка''. БракВРаботе} \\
\hline
\parbox[c][10mm]{45mm}{Balance} & \parbox{9mm}{593} & \parbox{9mm}{593} & \parbox{11mm}{1} & \parbox{10mm}{A} & \parbox{70mm}{Тип выработки заказа.} & \parbox{80mm}{} \\
\hline
\parbox[c][10mm]{45mm}{Sending of pallet} & \parbox{9mm}{594} & \parbox{9mm}{594} & \parbox{11mm}{1} & \parbox{10mm}{A} & \parbox{70mm}{Тип выпуска заказа.} & \parbox{80mm}{} \\
\hline
\parbox[c][10mm]{45mm}{Material handling line} & \parbox{9mm}{595} & \parbox{9mm}{596} & \parbox{11mm}{2} & \parbox{10mm}{A} & \parbox{70mm}{Код следующей линии.} & \parbox{80mm}{} \\
\hline
\parbox[c][10mm]{45mm}{Name of the box factory machine} & \parbox{9mm}{597} & \parbox{9mm}{611} & \parbox{11mm}{15} & \parbox{10mm}{A} & \parbox{70mm}{Наименование следующей линии.} & \parbox{80mm}{} \\
\hline
\parbox[c][10mm]{45mm}{Free} & \parbox{9mm}{612} & \parbox{9mm}{645} & \parbox{11mm}{34} & \parbox{10mm}{A} & \parbox{70mm}{Не используется.} & \parbox{80mm}{} \\
\hline

\caption{Структура блока сообщения о выполненных раскроях из SYNCRO в СИСТЕМУ}\label{tab:fosber_completed}
\end{longtable}  
\normalsize

\end{landscape} 
\normalsize

% ДАННЫЕ ОСТАНОВОВ

\scriptsize
\begin{landscape}
Структура блока сообщения об остановах из SYNCRO в СИСТЕМУ представлена в таблице \ref{tab:fosber_stoppage}.

ВНИМАНИЕ. Структура изменена.

\scriptsize

\begin{longtable}{|p{45mm}|p{6mm}|p{6mm}|p{8mm}|p{6mm}|p{70mm}|p{80mm}|}
\hline
\parbox[c][10mm]{45mm}{\centering Поле} & \parbox{6mm}{\centering С} & \parbox{6mm}{\centering По} & \parbox{8mm}{\centering Длина} & \parbox{5mm}{\centering Тип} & \parbox{70mm}{\centering Комментарий}  & \parbox{80mm}{\centering Opti-Corrugated} \\

\hline
\parbox[c][10mm]{45mm}{Code that cause of stoppage } & \parbox{9mm}{1} & \parbox{9mm}{4} & \parbox{11mm}{4} & \parbox{10mm}{A} & \parbox{70mm}{Код причины останова.} & \parbox{80mm}{Регистр ''Журнал работы оборудования''. Комментарий} \\
\hline
\parbox[c][10mm]{45mm}{Description that cause of stoppage} & \parbox{9mm}{5} & \parbox{9mm}{24} & \parbox{11mm}{20} & \parbox{10mm}{A} & \parbox{70mm}{Описание причины останова.} & \parbox{80mm}{Регистр ''Журнал работы оборудования''. Комментарий} \\
\hline
\parbox[c][10mm]{45mm}{Stopping area’s code} & \parbox{9mm}{25} & \parbox{9mm}{28} & \parbox{11mm}{4} & \parbox{10mm}{A} & \parbox{70mm}{Код области, в которой произошел останов.} & \parbox{80mm}{Регистр ''Журнал работы оборудования''. Комментарий} \\
\hline
\parbox[c][10mm]{45mm}{Description of stopping area} & \parbox{9mm}{29} & \parbox{9mm}{48} & \parbox{11mm}{20} & \parbox{10mm}{A} & \parbox{70mm}{Описание области, в которой произошел останов.} & \parbox{80mm}{Регистр ''Журнал работы оборудования''. Комментарий} \\
\hline
\parbox[c][10mm]{45mm}{Stoppage time} & \parbox{9mm}{49} & \parbox{9mm}{56} & \parbox{11mm}{8} & \parbox{10mm}{A} & \parbox{70mm}{Общее время простоя.} & \parbox{80mm}{} \\
\hline
\parbox[c][10mm]{45mm}{Start date } & \parbox{9mm}{57} & \parbox{9mm}{64} & \parbox{11mm}{8} & \parbox{10mm}{A} & \parbox{70mm}{Дата начала останова.} & \parbox{80mm}{Регистр ''Журнал работы оборудования''. Период} \\
\hline
\parbox[c][10mm]{45mm}{End date} & \parbox{9mm}{65} & \parbox{9mm}{72} & \parbox{11mm}{8} & \parbox{10mm}{A} & \parbox{70mm}{Дата окончания останова.} & \parbox{80mm}{Регистр ''Журнал работы оборудования''. Период} \\
\hline
\parbox[c][10mm]{45mm}{Start time} & \parbox{9mm}{73} & \parbox{9mm}{80} & \parbox{11mm}{8} & \parbox{10mm}{A} & \parbox{70mm}{Время начала останова.} & \parbox{80mm}{Регистр ''Журнал работы оборудования''. Период} \\
\hline
\parbox[c][10mm]{45mm}{End time} & \parbox{9mm}{81} & \parbox{9mm}{88} & \parbox{11mm}{8} & \parbox{10mm}{A} & \parbox{70mm}{Время окончания останова.} & \parbox{80mm}{Регистр ''Журнал работы оборудования''. Период} \\
\hline
\parbox[c][10mm]{45mm}{Order’s code} & \parbox{9mm}{89} & \parbox{9mm}{92} & \parbox{11mm}{4} & \parbox{10mm}{A} & \parbox{70mm}{Номер заказа.} & \parbox{80mm}{} \\
\hline
\parbox[c][10mm]{45mm}{Modifications} & \parbox{9mm}{93} & \parbox{9mm}{94} & \parbox{11mm}{2} & \parbox{10mm}{N} & \parbox{70mm}{Номер модификации.} & \parbox{80mm}{} \\
\hline
\parbox[c][10mm]{45mm}{Shift} & \parbox{9mm}{95} & \parbox{9mm}{95} & \parbox{11mm}{1} & \parbox{10mm}{A} & \parbox{70mm}{Номер смены.} & \parbox{80mm}{} \\
\hline
\parbox[c][10mm]{45mm}{Comments} & \parbox{9mm}{96} & \parbox{9mm}{295} & \parbox{11mm}{200} & \parbox{10mm}{A} & \parbox{70mm}{Комментарий, добавленный оператором} & \parbox{80mm}{Регистр ''Журнал работы оборудования''. Комментарий} \\
\hline
\parbox[c][10mm]{45mm}{Free} & \parbox{9mm}{296} & \parbox{9mm}{345} & \parbox{11mm}{50} & \parbox{10mm}{A} & \parbox{70mm}{Не используется.} & \parbox{80mm}{} \\
\hline

\caption{Структура блока сообщения об остановах из SYNCRO в СИСТЕМУ}\label{tab:fosber_stoppage}
\end{longtable}  
\normalsize

\end{landscape} 
\normalsize

%------------------------------------------------------------------

\subsection{Функциональные требования}

\subsubsection{Подключиться к SYNCRO}

При нажатии на кнопку ''Подключиться'' СИСТЕМА должна начать опрашивать SYNCRO с периодом, указанным в параметре ''ИнтервалОпроса'' в свойствах оборудования (справочник ''Оборудование''), указанного в документе выработки ''ОтчетПроизводстваЛГК''.

При подключении оборудования СИСТЕМА должна формировать строку команды для SYNCRO с командой ''VE'' для получения данных выработки.
Структура блока сообщения указана в таблице \ref{tab:fosber_completed}.
По каждой полученной строке СИСТЕМА попытается определить раскрой. Если он уже есть в документе, то для него обновятся параметры, заполнится факт выработки, дата и время окончания, брак. Если такого раскроя нет, то СИСТЕМА по идентификатору Run ID 2 добавит раскрой из общего списка запланированных заданий, а затем также обновит параметры и заполнит фактические значения. В противном случае, СИСТЕМА сформирует сообщение об ошибке. 

Затем СИСТЕМА сформирует команду ''SE'' для получения текущего статуса.
Структура блока сообщения указана в таблице \ref{tab:fosber_status}.
Если идентификатор текущего раскроя документа совпадает с ответом SYNCRO, то для этого раскроя обновятся текущие показатели. Таким образом, фактические данные будут обновляться в режиме реального времени. 

Если идентификатор раскроя из ответа SYNCRO отличается, тогда будут получены параметры текущего раскроя гофроагрегата командой ''LI''.
Структура блока сообщения указана в таблице \ref{tab:fosber_block}.
Далее СИСТЕМА определит, есть ли такой раскрой в документе выработки. Если есть - обновит его параметры, если нет - по идентификатору Run ID 2 добавит раскрой из общего списка запланированных заданий и обновит. Затем в документе изменится идентификатор текущего раскроя на текущий и продолжится чтение данных SYNCRO.

\subsubsection{Добавить раскрои в SYNCRO}

В форме редактирования документа ''ОтчетПроизводстваЛГК'' СИСТЕМА должна позволять выделять  несколько строк раскроев. 
По нажатию на кнопку ''Выгрузить раскрои'' СИСТЕМА должна формировать отдельную строку команды ''PE'' по каждому раскрою отдельно.
Структура блока сообщения указана в таблице \ref{tab:fosber_block}.

При получении команды ''ОК'' от SYNCRO СИСТЕМА должна установить признак ''Выгружен'' для выгруженного раскроя.
В противном случае СИСТЕМА должна выдавать сообщение об ошибке, полученное от SYNCRO.

\subsubsection{Обновить раскрои из SYNCRO}

СИСТЕМА должна автоматически обновлять параметры раскроя в течение сеанса, пока раскрой не будет выполнен на гофроагрегате.
Параметры получаются из SYNCRO по идентификатору Run ID 1. Если СИСТЕМА не находит его, то возвращает сообщение об ошибке, в противном случае данные найденного раскроя обновляются данными SYNCRO.

\subsubsection{Удалить раскрой в СИСТЕМЕ}

По нажатию на кнопку ''Удалить раскрой'' СИСТЕМА не должна позволять удалить раскрой, если в нем заполнен факт выработки. Если раскрой ещё не вырабатывался, то ему СИСТЕМА устанавливает признак ''Не загружен'' и должна удалить раскрой из документа. 

При удалении в СИСТЕМЕ необходимо посылать команду на удаление раскроя в SYNCRO 
''CA'' + ID раскроя.

Запустив список заказов, SYNCRO ищет один раскрой с тем же Run ID 1, что и указанный в поле ID данной команды. Если раскрой не найден, SYNCRO вернет сообщение об ошибке, в противном случае удаляет раскрой.

\subsubsection{Удалить раскрой в СИСТЕМЕ в конце смены}

При окончании смены по нажатию команды ''Завершить смену'' СИСТЕМА должна удалить раскрои в очереди на SYNCRO, которые остались выгруженными, но не выполненными.

При удалении в СИСТЕМЕ необходимо посылать команду на удаление раскроя в SYNCRO 
''CT'' + ID раскроя.
ID раскроя - первый в списке невыполенных раскроев из плана.
SYNCRO будет искать раскрой в очереди по ID. Если раскрой не найден, будет возвращено сообщение об ошибке. В другом случае SYNCRO удали все следующие за указанными раскрои в списке, начинающимся с найденного по ID раскрою.
SYNCRO не может удалить первый и второй раскрой в очереди.


\subsubsection{Удалить раскрой в SYNCRO}

При удалении раскроя на стороне SYNCRO раскрой должен удаляться из списка раскроев в SYNCRO. При обновлении данных в СИСТЕМЕ необходимо проверить список текущих раскроев. Удаленный на стороне SYNCRO раскрой должен быть помечен как "Не выгружен". 

% Удаление из SYNCRO должно производиться вручную.

\subsubsection{Закрыть раскрой}

СИСТЕМА должна отслеживать состояние по текущему раскрою с помощью команды ''SE''.
При изменении идентификатора Run ID 1 СИСТЕМА должна закрывать текущий раскрой, устанавливать признак ''Закрыт'', обновлять параметры, время и заполнять фактические данные выработки. Затем СИСТЕМА переключается на новый раскрой с идентификатором, полученным от SYNCRO. Порядок выполнения раскроев на Fosber может быть изменен и отличаться от порядка в документе ''ОтчетПроизводстваЛГК''. 

\subsubsection{Получать данные по остановам}

СИСТЕМА должна получать данные по остановам гофроагрегата с периодичностью, указанной в параметре ''ИнтервалПолученияСпискаОстановов'' в свойствах оборудования (Справочник ''Оборудование''), указанного в документе выработки ''ОтчетПроизводстваЛГК''.

По каждой строке данных СИСТЕМА формирует две записи в регистре ''Журнал работы оборудования'' о начале и окончании останова. В поле ''ПричинаОстанова'' для записи устанавливается значение ''Причина не определена'', и оператор
% (или мастер смены) 
должен самостоятельно указать причину останова, выбрав ее из списка справочника ''Причины остановов''. В поле ''Комментарий'' СИСТЕМА запишет данные о проблемном месте, из-за которого произошёл останов, и описание причины.

Коды причин простоев Syncro должны совпадать с кодами причин простоев в СИСТЕМЕ.

\subsubsection{Отключиться от SYNCRO}

По команде отключения СИСТЕМА должна перестать опрашивать SYNCRO и получать данные.
При закрытии формы документа опрос также должен быть прекращен.


