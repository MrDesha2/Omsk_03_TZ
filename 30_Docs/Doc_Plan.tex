\subsection{Документ <<План>>}
\label{doc:Plan}

\subsubsection{Описание предметной области}

Документ предназначен для составления планов работы гофроагрегата и перерабатывающего оборудования.

Документ существует в СИСТЕМЕ, необходимо внести изменения в функциональность.

При использовании механизма автоматического планирования работы гофроагрегата на предприятии увеличится количество раскроев, что приведет к разделению выхода паллет после гофроагрегата по одному заказу. 
Это потребует  организовать хранение паллет после гофроагрегата для удобного поиска паллет одного заказа. 

\subsubsection{Функциональные требования}

\point{Позаказный учет}

При внедрении СИСТЕМЫ ПРЕДПРИЯТИЕ должно перейти на планирование заданий и внесение выработки только с использованием номеров заказов менеджеров, не допускается (и невозможно) выдавать задания на номенклатуру без указания номера заказа.
Соответственно во многих формах и отчетах СИСТЕМЫ выводится не только номенклатура, но и номер заказа менеджера.

\point{Форма редактирования. Форма Редактирование раскроев}

В табличной части документа план изменить расположение колонок: 
Длина, ширина, рилевки.

\point{Форма редактирования. Вкладка ''Планирование линий''}

В табличной части документа план изменить расположение колонок: 
Длина, ширина.



\point{Печатная форма задания на гофроагрегат}

% Внести изменения в во внешний вид текущего задания, печатаемого из СИСТЕМЫ.
% \begin{itemize}
% \item Убрать колонку <<шт. изделий>>;
% \item Убрать колонку <<требуется изделий>>;
% \item Убрать колонку <<раскрой>>;
% \item Колонку <<Ширина>> заменить на <<Формат (полотна)>>;
% \item <<п.п>> (погонные метры) заменить на <<м2>>;
% \item переименовать <<\% потерь>> на <<\% обрези>>.
% \end{itemize}

% В печатном задании на ГА объем сырья надо учитывать с допуском (использовать коэффициенты гофрирования с браком из справочника ''Профили'').


%\point{Учет покупных заготовок}
%
%
%При построении раскроев уменьшать количество заготовок у заказов на величину, указанную в атрибуте ''КоличествоПокупныхЗаготовок''.
%Объем для кроя должен быть взят за минусом заготовок, которые пользователь с ролью !Плановик решил заказать на стороне и не выпускать на гофроагрегате.


