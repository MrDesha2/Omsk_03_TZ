\subsection{Обработка <<Непрерывный план>>}
\label{doc:PlanInfinite}

\subsubsection{Функциональные требования}

% Добавить из Архбум

\point {Гофроагрегат. Подсветка заданий}

Добавить подсветку раскроев в форме непрерывного плана.
Подсвечивать строки как для раскроев с разными типами рилевок для сделующих случаев.
Менее значения параметра «Минимальное расстояние между рилевками» одно из следующего:
\begin{itemize}
    \item 
 расстояние между рилевками внутри изделия (проверяется для каждого заказа раскроя)
\item  Сумма значений крайних рилевок двух разных заказов из раскроя (если в раскрое есть два заказа). Пусть рилевки первого заказа A1/A2/A3, а рилевки второго заказа B1/B2/B3, тогда надо проверить A1+B3 и A3+B1
\item   Двойное значение каждой из крайних рилевок, если заказ в раскрое идет более чем одной полосой. Для обозначений выше это означает проверку 2*A1 и 2*A
\end{itemize}

% \point {Отображение проблем по заказам}

% Добавить вывод колонки <<Проблемы>>, ячейка которой должна быть закрашена красным цветом, если для данного заказа имеются записи в регистре <<Проблемы по заказам>> --- это будет сигналом для плановика, что необходимо открыть форму заказа и просмотреть список проблем. 


\point{Исправить проверку выполнения задания}

В непрерывном плане при проверке выполнения данного задания, должна проверять  выполнение задания из значения ''Процент выпуска'' технологической карты заказа , а в случае если полученное значение равно «0» брать значение по умолчанию из настроек Системы.


\point{Добавить команду разделения заказов}

В форме «Непрерывный план» для табличной части плана по линиям переработки в контекстное меню необходимо добавить функцию разделения задания. При выборе пользователем данной команды должна быть открыта дополнительная форма, в которой пользователь должен указать объем, остающийся в первой части задания, а Система должна автоматически рассчитать количество для отделяемой части как разницу между объемом задания и объемом первой части.


\point{Печатать форму ТК}

Изменить вызов печатной формы в ТК в таблице заказов на печать полной формы ТК.
